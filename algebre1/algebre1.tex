\documentclass[a4paper,10pt]{article}

\usepackage[utf8x]{inputenc}
\usepackage{amsfonts}
\usepackage{amsmath}
\usepackage[french]{babel}
\usepackage{xspace}
\usepackage{lmodern}
\usepackage[babel]{microtype}
\usepackage[T1]{fontenc}
\usepackage{graphicx}
\usepackage{pst-all}
\author{David Hauweele}
\title{Synthèse Algèbre I}
\date{2007-10-23}

\newcommand{\card}[1]{\left\vert #1 \right\vert}
\newcommand{\Fix}{\mbox{Fix }}
\newcommand{\Sgn}{\mbox{Sgn }}
\newcommand{\Supp}{\mbox{Supp }}
%%\newcommand{\mod}{\mbox{ mod }}
\newcommand{\dom}{\mbox{Dom }}
\newcommand{\but}{\setminus}
\newcommand{\Ker}{\mbox{Ker }}
\newcommand{\im}{\mbox{Im }}
\newcommand{\grp}[1]{\left\langle #1 \right\rangle}
\newcommand{\ap}{\rightarrow}
\newcommand{\nap}{ \longrightarrow {\kern-13,5pt{\backslash}}~~}
\newcommand{\apw}[1]{\stackrel{#1}{\rightarrow}}
\newcommand{\C}{\mathbb{C}}
\newcommand{\I}{\mathbb{I}}
\newcommand{\R}{\mathbb{R}}
\newcommand{\N}{\mathbb{N}}
\newcommand{\Z}{\mathbb{Z}}
\newcommand{\so}{\Rightarrow}
\newcommand{\os}{\Leftarrow}
\newcommand{\ioi}{\Leftrightarrow}
\newcommand{\tset}[1]{\left\lbrace #1 \right\rbrace}
\newcommand{\conv}[1]{\mathop{\longrightarrow}\limits_{#1}}
\newcommand{\nconv}[1]{\mathop{\nap}\limits_{#1}}
\newcommand{\abs}[1]{\left\vert #1 \right\vert}
\newcommand{\ceil}[1]{\left\lceil #1 \right\rceil}
\newcommand{\normal}{\triangleleft}
\newcommand{\PGCD}{\mbox{PGCD}}
\newcommand{\PPCM}{\mbox{PPCM}}
\newcommand{\ordre}{\mbox{ordre }}
\newcommand{\Inn}{\mbox{Inn }}
\begin{document}

\maketitle
Last revision : \today

%Document réalisé en \LaTeX

%Document réalisé sous Vim

%Figures réalisées sous XFig

%Le tout sur Linux et FreeBSD
\tableofcontents
\newpage

\section{Introduction}

\subsection{Ensemble}

Un ensemble est une collection d'objet (l'ordre importe peu). On peut citer :

$$\mathbb{Z},\mathbb{R}^2,\mathbb{C}^2$$

\subsection{Element}

Un objet de la collection formant un ensemble.

Pour signifier qu'un élément appartient à un ensemble on utilise $\in$ (appartient). Par exemple $2 \in \mathbb{Z}$.

\subsection{Sous-ensemble ou partie}

On dit que $Y$ est un sous-ensemble de $X$ ou aussi une partie\footnote{Ne pas confondre partie et élément, en effet $x\in X$ et $\left\lbrace x \right\rbrace \subset X$.} de $X$ si tous les éléments de $Y$ sont éléments de $X$

Pour signifier qu'un ensemble est sous-ensemble d'un autre ensemble on utilise $\subset$ (inclu), $\subseteq$ (inclu ou égaux). Par exemple $\mathbb{Z} \subset \mathbb{R}$.

\subsection{Ensemble des parties}

On considère ici un ensemble qui comprend toutes les parties d'un autre ensemble.

$$\forall Y \subseteq X, Y \in \mathcal{P} X$$

On peut aussi considérer un ensemble qui comprend tous les ensemble des parties d'un ensemble.

$$\forall Y \subseteq \mathcal{P} X, Y \in \mathcal{P}^2 X$$

Prenons par exemple l'ensemble $\mathbb{R}^2$, on peut dire qu'une droite $D$ est un sous-ensemble tel que $D \subset \mathbb{R}^2$ donc $D \in \mathcal{P} \mathbb{R}^2$. Prenons maintenant deux droites $D_1$ et $D_2$, on peut dire que $D_1,D_2 \in \mathcal{P} \mathbb{R}^2$ mais on peut aussi dire que l'ensemble $\left\lbrace D_1 ; D_2  \right\rbrace \subset \mathcal{P} \mathbb{R}^2$ et encore $\left\lbrace D_1 ; D_2  \right\rbrace \in \mathcal{P}^2 \mathbb{R}^2$

\newpage


\subsection{Union d'ensembles}

On note $A \cup B$ l'ensemble constitué des éléments de $A$ et des éléments de $B$

$$A \cup B = \tset{x \vert x\in A \mbox{ ou } x \in B}$$

\subsection{Intersection d'ensembles}

On note $A \cap B$ l'ensemble constitué des éléments de $A$ ou $B$

$$A \cap B = \left\{ x \vert x \in A \mbox{ et } x \in B \right\}$$

\subsection{Partition}

Une partition c'est une famille $\mathcal{F}$ de parties de $X$ qui vérifie:

\begin{itemize}
 \item{On recouvre $X$}
 \item{Tous les éléments de $\mathcal{F}$ sont non vides}
 \item{Deux éléments de $\mathcal{F}$ sont soit égaux, soit disjoints}
\end{itemize}

Autrement dit :

$$\bigcup \mathcal{F} = X \mbox{ ou encore } \bigcup\limits_{Y \in \mathcal{F}} Y = X$$

$$\forall Y (Y \in \mathcal{F} \Rightarrow Y \neq \emptyset)$$

$$\forall Y_1, Y_2 (Y_1 \in \mathcal{F} \wedge Y_2 \in \mathcal{F} \Rightarrow (Y_1 = Y_2) \vee (Y_1 \cap Y_2 = \emptyset))$$

\subsection{Relation binaire}

Dans une relation binaire (par exemple $\mathcal{R}$) on met en relation des couples d'éléments d'un ensemble donné (par exemple $x \mathcal{R} y$), une relation binaire sur $X$ est donc une partie de $X^2$.

\subsection{Relation d'équivalence}

Une relation binaire sur $X$ est une relation d'équivalence si elle est réfléxive, symétrique et transitive.

Autrement dit :

$$\forall x \in X, x \mathcal{R} x$$
$$\forall x,y \in X, x \mathcal{R} y \Leftrightarrow y \mathcal{R} x$$
$$\forall x,y,z \in X, (x \mathcal{R} y \wedge x \mathcal{R} z) \Rightarrow y \mathcal{R} z$$

On note souvent une relation d'équivalence par $\sim$ ou aussi $\equiv$.

\subsection{Classe d'équivalence}

On considère $\left\lbrace y \vert x \sim y \right\rbrace$, appelé classe d'équivalence et souvent noté $\left[ x \right]_\sim$

La famille $\mathcal{F} = \left\lbrace { \left[ x \right]_\sim } \vert x \in X \right\rbrace$ est une partition sur $X$.

Pour une partition $\mathcal{F}$ on peut trouver une relation d'équivalence.

\subsection{Quotient}

Soit $f : A \rightarrow B$ une fonction quelconque.

Soit une relation d'équivalence $\sim$ telle que $x \sim y \Leftrightarrow f(x) = f(y)$

On note $f^{-1}b = \left\lbrace x \in A \vert f(x) = b \right\rbrace$

Premièrement on a $\left[ a \right]_\sim = f^{-1} (f(a))$

Deuxièmement on a $\exists \mathcal{G} : \mathcal{F} \rightarrow \mbox{ Im } f : \left[ a \right]_\sim \mapsto f(a) \mbox{ ou encore } f^{-1} b \mapsto b$ 

%\section{Théorie des groupes}
\subsection{Modulo}
L'opération de modulo par exemple $n$ modulo $d$ (noté aussi $n \mod d$) c'est le reste\footnote{Il est également utile de donner une interprétation plus géométrique de cette opération. On peut par exemple s'imaginer un angle restreint à un interval $2 \pi$ par l'opération modulo.} de la division de $n$ par $d$.
Autrement dit, soit $n = qd + r$ (algorithme d'Euclide) alors $n \mod d = n - qd = r$.
Notons que $0 \leq r < d$, le résultat de l'opération modulo doit donc lui appartenir à cette intervale.

\subsection{Congruence}

La congruence signifie avoir le même reste modulo $d$, souvent noté $\equiv_d$.

$$x \equiv_d y \Leftrightarrow x \mod d = y \mod d \Leftrightarrow \exists q : x - y = qd$$
\subsection{Ecriture en base $n$}
Prenons l'exemple de la base 17. Le nombre 2597 peut s'écrire en base 17 comme $8.17^2 + 2597 \mod 17^2 = 8.17^2 + 285$ mais $285$ peut s'écrire comme $16.17^1 + 285 \mod 17^1 = 16.17^1 + 4$ et $4$ peut s'écrire comme $4.17^0 + 4 \mod 17^0 = 4.17^0 + 1$. Autrement dit, $2597 = 8.17^2 + 16.17^1 + 4.17^0$. Si on définie les chiffres de la base dix-sept par $\left[ 0, 1, 2, 3, 4, 5, 6 , 7, 8, 9 , A, B, C, D, E, F, G \right]$ alors 2597 s'écrit en base 17 : $800 + F0 + 4 = 8F4$.

D'une manière plus générale pour écrire dans la base $n$, soit $r_j$ la valeur du $j-\mbox{ième}$ rang et $m$ le nombre à écrire. Alors le chiffre au $k-\mbox{ième}$ rang est donné par $\alpha_k = (m - r_{k-1}.\alpha_{k-1}) \mbox{ div } r_k$ autrement dit : 
$$\alpha_k = (m \mod r_{k+1}) \mbox{ div } r_k$$

Notons que les chiffres de l'écriture en base $n$ appartiennent à l'intervalle définie par $\left[ \mbox{min } \alpha , \mbox{Max } \alpha \right] \subset \mathbb{Z}$.

Notons aussi que le rang ne se compte pas à partir de 1 mais à partir de 0, par exemple le nombre en base 16 EA28, le rang 0 est 8, le rang 1 est 2, le rang 3 est A et le rang 4 est E.
\subsection{Suite de Fibonacci}
Soit la suite définie par $F_{n+2} = F_n + F_{n+1}$ avec $F_0$ et $F_1$ donné\footnote{Une suite définie par $F_{n+k+1} = \sum_{i=n}^{n+k} F_{n+i}$ avec $F_0$ à $F_k$ donné est une suite dite de K-bonacci.}. On dit que la suite $F$ est de Fibonacci.

La suite de Fibonacci a une croissance exponentielle, elle est lié au nombre d'or. En effet $\varphi = \lim\limits_{n \rightarrow \infty} \frac{\vert F_{n+1} \vert}{\vert F_n \vert}$

\newpage

\subsection{Ecriture en base de Fibonacci}
La valeur au $n-\mbox{ième}$ rang de l'écriture en base de Fibonacci est donnée par le $n-\mbox{ième}$ terme de la suite de Fibonacci à savoir $F_n$. Le chiffre au rang $k$ est donc :
$$\alpha_k = (m \mod F_{k+1}) \mbox{ div } F_k$$

L'écriture en base de Fibonacci est utilisée pour montrer la complexité de l'agorithme d'Euclide.

\subsection{Notation}

On note $\mathbb{Z} / x \mathbb{Z}$ l'ensemble $\left\lbrace 0, 1, ... , x \right\rbrace$ et $\mathbb{F}_{x}$ si $x$ est premier.

\subsection{Calcule}

Prenons par exemple $2^{1993} \mod 5 = 2 . 2^{1992} \mod 5 = 2 . \left( 2^4 \right)^{498} \mod 5$ mais on sait que $2^4 \mod 5 = 1$ donc $2 . \left( 2^4 \right)^{498} \mod 5 = 2 . 1^{498} \mod 5 = 2$.

\subsection{Groupe}

Un groupe est un ensemble munie d'une loi qui est partout définie et interne, contient un neutre, contient l'inverse de chaque élément, est associatif.

Exemple :

$$\forall x,y \in \mathbb{F}_5, x +_5 y \in \mathbb{F}_5$$
$$\forall x \in \mathbb{F}_5, x +_5 0 = 0 +_5 x = x$$
$$\forall x \in \mathbb{F}_5, \exists y \in \mathbb{F}_5 : x +_5 y = 0$$
$$\forall x,y,z \in \mathbb{F}_5, (x +_5 y)  +_5 z  = x +_5 (y +_5 z)$$

On dira que $\left\langle \mathbb{F}_5 , +_5 , 0 \right\rangle$ est un groupe. De plus, on peut dire que c'est un groupe commutatif car :

$$\forall x,y \in \mathbb{F}_5, x +_5 y = y +_5 x$$

\newpage 

\section{Théorie des groupes}

\subsection{Opération binaire}

Une opération binaire $\star$ sur un ensemble $G$ est une fonction 

$$\star : G \times G \rightarrow G : (x,y) \mapsto x \star y$$

\subsection{Monoïde}

Un monoïde ou semi-groupe avec neutre c'est la donnée\footnote{Cette donnée est notée $\left\langle G , \cdot_G \right\rangle$ ou $\left\langle G , \cdot_G , 1_G \right\rangle$. Reste valable pour les monoïdes, groupes et groupes commutatifs.} d'un ensemble $G \neq \emptyset$ et d'une opération binaire notée $\cdot_G$ qui vérifie :

\begin{itemize}
 \item[1]{L'opération $\cdot_G$ est interne}
 \item[2]{L'opération $\cdot_G$ est partout définie}
 \item[3]{L'opération $\cdot_G$ est associative}
 \item[4]{L'opération $\cdot_G$ admet un neutre notée $1_G$}
\end{itemize}

Autrement dit :

$$\mbox{(1) et (2) }\forall x,y \in G, x \cdot_G y \in G$$
$$\mbox{(3) } \forall x,y,z \in G, ( x \cdot_G y ) \cdot_G z = x \cdot_G ( y \cdot_G z )$$
$$\mbox{(4) } \forall x \in G, 1_G \cdot_G x = x \cdot_G 1_G = x$$

\subsection{Groupe}

Un groupe c'est la donnée d'un ensemble $G \neq \emptyset$ et d'une opération binaire notée $\cdot_G$ qui vérifie :

\begin{itemize}
 \item[1]{L'opération $\cdot_G$ est interne}
 \item[2]{L'opération $\cdot_G$ est partout définie}
 \item[3]{L'opération $\cdot_G$ est associative}
 \item[4]{L'opération $\cdot_G$ admet un neutre notée $1_G$}
 \item[5]{L'opération $\cdot_G$ est inversible}
\end{itemize}

Autrement dit :

$$\mbox{(1) et (2) }\forall x,y \in G, x \cdot_G y \in G$$
$$\mbox{(3) } \forall x,y,z \in G, ( x \cdot_G y ) \cdot_G z = x \cdot_G ( y \cdot_G z )$$
$$\mbox{(4) } \forall x \in G, 1_G \cdot_G x = x \cdot_G 1_G = x$$
$$\mbox{(5) } \forall x \in G, \exists y \in G : x \cdot_G y = y \cdot_G x = 1_G$$

\newpage

\subsection{Groupe abélien}

Un groupe abélien (ou groupe commutatif) c'est la donnée d'un ensemble $G \neq \emptyset$ et d'une opération binaire notée $\cdot_G$ qui vérifie :

\begin{itemize}
 \item[1]{L'opération $\cdot_G$ est interne}
 \item[2]{L'opération $\cdot_G$ est partout définie}
 \item[3]{L'opération $\cdot_G$ est associative}
 \item[4]{L'opération $\cdot_G$ admet un neutre notée $1_G$}
 \item[5]{L'opération $\cdot_G$ est inversible}
 \item[6]{L'opération $\cdot_G$ est commutative}
\end{itemize}

Autrement dit :

$$\mbox{(1) et (2) }\forall x,y \in G, x \cdot_G y \in G$$
$$\mbox{(3) } \forall x,y,z \in G, ( x \cdot_G y ) \cdot_G z = x \cdot_G ( y \cdot_G z )$$
$$\mbox{(4) } \forall x \in G, 1_G \cdot_G x = x \cdot_G 1_G = x$$
$$\mbox{(5) } \forall x \in G, \exists y \in G : x \cdot_G y = y \cdot_G x = 1_G$$
$$\mbox{(5) } \forall x,y \in G, x \cdot_G y = y \cdot_G x$$

\subsection{Groupe linéaire de dimension $n$}

Le groupe linéaire de dimension $n$ (noté $GL_n(\mathbb{R})$) c'est l'ensemble des matrices $n \times n$ inversibles à coefficients dans $\mathbb{R}$.

\subsection{Ensemble des bijections de $A$ vers $B$}

On note l'ensemble des bijections de $A$ vers $B$ $\mbox{Bij} (A,B)$.

Notons que la composition de fonction est définie seulement si $A=B$.

\subsection{Sous groupe}
Soit $\grp{G,._G,1_G}$ un groupe et $H \subset G$, on dit que $H$ est un sous-groupe de $G$ (on note $\grp{H,._G,1_G} < \grp{G,._G,1_G}$ ou $H < G$) si:

\begin{itemize}
 \item[1]{Le neutre $1_G$ appartient à $H$}
 \item[2]{L'ensemble $H$ forme un groupe avec l'opération $._G$ et le neutre $1_G$}
\end{itemize}

\subsection{Critère de sous groupe}
Soit $\grp{G,._G,1_G}$ un groupe. Soit $H \subset G$ alors $H$ est un sous-groupe de $G$ ssi :

\begin{itemize}
 \item[1]{Le neutre $1_G$ appartient à $H$}
 \item[2]{L'opération $._G$ et interne à $H$}
 \item[3]{L'inverse de tout élément de $H$ (au sens de $G$) est élément de $H$}
\end{itemize}

\subsection{Morphisme}

Soit $\grp{G,._G,1_G}$ et $\grp{H,._H,1_H}$ deux groupes quelconques, soit $\sigma$ une application $G \ap H$. On dit que $\sigma$ est un morphisme de groupe si :

$$\forall g_1, g_2 \in G, \sigma(g_1) ._H \sigma(g_2) = \sigma(g_1 ._G g_2)$$

\subsection{Monomorphisme}

Un monomorphisme est un morphisme injectif.

\subsection{Epimorphisme}

Un épimorphisme est un morphisme surjectif.

\subsection{Isomorphisme}

Un isomorphisme est un morphisme bijectif (càd un monomorphisme et un épimorphisme).

\subsubsection{Remarque sur les morphismes}
Soit $\sigma : G \ap H$ un morphisme de groupes,

Alors $\sigma (1_G) = 1_H$

Alors $\forall g \in G, \sigma(g^{-1}) = \sigma(g)^{-1}$

\subsection{Domaine}

Soit $f: A \ap B$, on a $\dom f = \tset{a \in A, f(a) \mbox{ existe}}$

\subsection{Ensemble image}

Soit $f: A \ap B$, on a $\im f = \tset{b \in B \vert \exists a, f(a)=b} = \tset{f(a) \vert a \in \dom f}$

Dans le cas d'un morphisme de $G$ vers $H$ on a $\im \sigma < \grp{H, . , 1_H}$

\subsection{Ensemble préimage}

Soit $f:A \ap B$, on a $f^{-1}(b) = \tset{a \in A \vert f(a) = b}$ càd tous les points qui atteignent $b$.

Note $f^{-1} \neq \emptyset \ioi b \in \im f$

Note $\left[ a \right]_{\sim_f} = f^{-1} (f(a))$

Note $\bigcup f^{-1}B = A$ et $b_1 \neq b_2 \ioi f^{-1}(b_1) \neq f^{-1}(b_2)$


\newpage

\subsection{Noyau}

Soit $f:A \ap B$, on a $\Ker f = f^{-1}(0) = \tset{a \in A \vert f(a) = 0}$

Soit $\sigma: G \ap H$ un morphisme, on a $\Ker \sigma = \sigma^{-1}(0) = \tset{a \in A \vert \sigma(a) = 1_H}$

Dans le cas d'un morphisme, on a aussi $\Ker \sigma < \grp{G, . , 1_G}$ 

\begin{figure}[h]
	\centering{\input{kersigma.pstex_t}}
\end{figure}
 
\subsection{Surjectivité, Injectivité, Bijectivité}

Soit $f : A \ap B$ une fonction,

$f$ est surjective sur $B$ ssi $\forall b, f^{-1}(b) \neq \emptyset$

$f$ est injective sur $\dom f$ ssi $f^{-1}(b)$ est un singleton ou vide (pas garantie qu'on à la surjection).

$f$ est une bijection de $\dom f$ sur $B \ioi \forall b \in B, f^{-1}(b)$ est un singleton. 

\subsection{Epimorphisme, Monomorphisme,Isomorphisme}

Soit $\sigma : G \ap H$ un morphisme de groupe,

$\sigma$ est épimorphisme ssi $\forall b, \sigma^{-1}(b) \neq \emptyset$

$\sigma$ est monomorphisme ssi $\sigma^{-1}(b)$ est un singleton ou vide.

$$\Updownarrow$$

$\sigma$ est monomorphisme ssi $\Ker \sigma = \tset{1_G} \ioi \mbox{L'équation } \sigma(g) = 1_H \mbox{ a comme unique solution } 1_G$

\subsection{Même images}

Dans un morphisme, deux points $x$ et $y$ ont même image ssi $xy^-1 \in \Ker \sigma$

\newpage

\subsection{Lemme}

Soit $\sigma : G \ap H$ un morphisme de groupe.

Soit $g \in G$ et soit $h \in H$ définie par $\sigma(g) = h$

Alors $\sigma^{-1}(h) = g . \Ker{\sigma} \stackrel{\Delta}{=} \tset{g.u \vert u \in \Ker \sigma}$

\begin{figure}[h]
	\centering{\input{kerlemma.pstex_t}}
\end{figure}

\subsection{Bijection canonique}

On appelle bijection canonique la bijection 

$$\tau : \tset{\sigma^{-1}(h)\vert h \in \im \sigma} \apw{\tau} \im \sigma$$
$$\tau(\sigma^{-1}(h))=h$$
$$\tau(g.\Ker \sigma)=\sigma(g)$$

\subsection{Quotient de G par le noyau}

On note $\tset{\sigma^{-1} (h) \vert h \in \im \sigma}$ le quotient de $G$ par $\Ker \sigma$, noté plus souvent $\tset{g . \Ker \sigma \vert g \in G}$
%On note $\tset{ \sigma^{-1} \vert h \in \im \sigma}$ le quotient de $G$ par $\Ker \sigma$, note plus souvent $\tset{g.\Ker \sigma \vert g \in G}$

\subsection{Classe latérale}

On appelle aussi $g.\Ker \sigma$ une classe latérale car c'est une translation du noyau

\subsection{Multiplication des classes}

On définie la multiplication des classes par $(g_1 . \Ker \sigma) . (g_2.\Ker \sigma) = (g_1.g_2) . \Ker \sigma$

\subsection{Théorème fondamental, premier théorème d'isomorphisme}

Le quotient de $G$ par le noyau munie de la multiplication des classes est un groupe

La bijection canonique est un isomorphisme de groupe

\subsection{Isomorphe}

Soit $G$ et $H$ deux groupes

On dit $G$ est isomorphe à $H$ ssi il existe un isomorphisme de $G$ vers $H$

Ce qu'on note $G \cong H$

Remarque : $G \cong H$ est une relation d'équivalence

\subsection{Lemme}

$$g.\Ker \sigma.g^{-1} = \tset{g.x.g^{-1} \vert x \in \Ker \sigma} = \Ker \sigma$$

\subsection{Sous groupe normal}

On dit que $H < G$ est sous groupe normal (distingué) noté $H \normal G$ ssi 

$$\forall g \in G, g H g^{-1} = H$$
$$\Updownarrow$$
$$gH = Hg$$

On appelle $gH$ l'ensemble des classes latérale à gauche dans $G$

On appelle $Hg$ l'ensemble des classes latérale à droite dans $G$

Remarque :

On appelle $g H g^{-1}$ le conjugué de $H$

Remarque : 

Pour $\sigma: G \ap H, \Ker \sigma \normal G$

Si $G$ est commutatif alors $\forall H < G, H \normal G$

\subsection{Lemme d'égalité des classes}

$xH = yH$ ssi $x^{-1}y \in H$

\subsection{Quotient}

On note le quotient de $G$ par $H$:

$$G/H = \tset{gH \vert g \in G}$$

Remarque :

Le quotient forme une partition de $G$

\subsection{Groupe quotient}

$\grp{G/H,.,1}$ est un groupe appelé groupe quotient

\subsection{Epimorphisme canonique sur le quotient}

$\kappa : G \ap G/H : g \mapsto gH$ est un épimorphisme de groupe appelé épimorphisme canonique sur le quotient

Remarque :

$\Ker \kappa = H$

$\forall H \normal G, H = \Ker \kappa$

\subsection{Union des près-images (exercice du premier semestre)}

Soit $\sigma : G_1 \ap G_2$ un morphisme de groupe.

Soit $E \subseteq G_2$

On note $\sigma^-1 (E) = \tset{g \in G_2 \vert \sigma(g) \in E} = \bigcup\limits_{h \in E} \sigma^{-1}(h)$ avec $h \in E$

Soit $H < G_2$, alors $\Ker \sigma \subset \sigma^{-1}(H)$ et $\sigma^{-1}(H) < G_1$

Soit $L < G_1$ alors $\sigma(L) = \tset{\sigma(l) \vert l \in L} < \im \sigma = G_2$

\subsection{Corrolaire du théorème fondamental}
Comme $G / \Ker \sigma$ est isomorphe à $\im \sigma$, on a $\card{G} = \card{\im \sigma} . \card{\Ker \sigma}$

De plus si $G$ est fini on a $\card{\Ker \sigma}$ divise $\card{G}$ et $\card{\im \sigma}$ divise $\card{G}$


\subsection{Centre}

Soit $\grp{G,.,1}$ un groupe.

On définie le centre de $G$ l'ensemble des éléments qui commutent avec tous les autres

On note $Z(G) = \tset{g \in G \vert \forall h \in G, g.h = h.g}$

\section{Groupes finis et ordre}

\subsection{Action de $\Z$}

On a un morphisme $\sigma_g : \grp{\Z, +, 0} \ap G : z \mapsto g^z$

\subsection{Groupe cyclique, sous groupe engendré}

On apelle sous-groupe de $G$ engendré par $g$ ou groupe cyclique engendré par $g$ le plus petit sous-groupe de $G$ contenant $g$ 

On le note $\grp{g}$ 

On a $\grp{g} = \bigcap\limits_{g \in H < G} H$ 

On a $\grp{g} = \tset{g^z \vert z \in \Z}$

\subsection{Ordre d'un élément}

Soit $G$ un groupe $\grp{G,.,1}$ soit $g\in G$ l'ordre de $g$ dans $G$ et le plus petit entier non nul $n$ tel que :

$g^n = 1$ (dans le cas d'un groupe additif tel que $ng = 0$)


\subsection{Théorème de Lagrange}

Soit $\grp{G,.,1}$ un groupe fini, soit $H < G$

On a $\card{H}$ divise $\card{G}$

En particulier, si $g \in G$ on a $\ordre(g)$ divise $\card{G}$

\subsection{Théorème}

Si $G < \grp{\Z,+,0}$ alors $\exists n \in \N : G = n \Z$

\subsection{Fonction caractéristique}

Soit $X \subseteq Y$ un ensemble. 

La fonction caractéristique $1_X : Y \ap \tset{0,1} : y \mapsto 1_X(y)$

$1_X(y) = 1$ si $y \in X$

$1_X(y) = 0$ sinon

\subsection{Représentation binaire d'une partie de $\N$}

Soit $X \subseteq \N$
 
Posons $(x_n)_{n \in \N}$ définie par $x_n = 1_X(n)$ avec $1_X$ la fonction caractéristique de $X \subseteq \N$

On a les éléments $1_X(0),1_X(1),1_X(2),\cdots$ càd par exemple $0,1,0,\cdots$ qui représentent une branche infinie d'un arbre binaire.

Il y a une bijection entre l'ensemble des branches infinies de l'arbre et l'ensemble des parties de $\N$

On ne peut pas numéroter ces branches, elles ne sont pas dénombrables par $\N$

Ces branches sont dénombrables par $\R$

\subsection{Dénombrable}

L'ensemble des parties est dénombrable si la fonction $\sigma : \N \ap$ l'ensemble des branches est surjective

Dans le cas de $\N$ une telle fonction n'existe pas (paradoxe de Russel)

\subsection{Branche antidiagonale}

On définie la branche $AD_\sigma$ qui va à gauche au niveau $n$ ssi $\sigma(n)$ va à droite au niveau $n$

\subsection{Ensemble des parties de $\N$}


L'ensemble des parties/branches de $\N$ est noté $2^\N$

\subsection{Quantificateurs}

Comme quantificateurs on a $\forall$ et $\exists$

On a $\forall P = \neg \exists \neg P$

\subsection{Projection d'un ensemble}

Soit $E = \tset{(r_1,r_2) \vert \theta(r_1,r_2)} \subseteq V^2$

On a :

$P_x(E) = \tset{r_1 \in V \vert \exists r_2 \in V, \theta(r_1,r_2)}$ est la projection de $E$ sur l'axe $x$

$P_y(E) = \tset{r_2 \in V \vert \exists r_1 \in V, \theta(r_1,r_2)}$ est la projection de $E$ sur l'axe $y$

Remarque : les propriétés topologiques ne sont plus conservés par la projection. Par exemple le fait d'avoir un trou (comme le tore).

\subsection{PGCD}

Le $\PGCD(a,b)$ est le plus grand diviseur commun à $a$ et $b$

\subsection{PPCM}

Le $\PPCM(a,b)$ est le plus petit commun multiple

Remarque $\PPCM(a,b) \leq a.b$

\subsection{Remarque}

$\PPCM(a,b).\PGCD(a,b) = a.b$

\subsection{Diviseurs communs}

L'ensemble des diviseurs communs à $a$ et $b$ c'est $D_{a,b} = \tset{d \in \N \vert d \mbox{ div } a \mbox{et} d \mbox{ div } b}$

\subsection{Multiples communs}

L'ensemble des multiples communs à $a$ et $b$ c'est $M_{a,b} = \tset{m \in \N \vert m \mbox{ est mul. de } a \mbox{ et } m \mbox{ est mul. de } b}$

\subsection{Méthodes pour trouver le PGCD}

\subsubsection{Décomposition en facteurs premiers}

On décompose $a$ en un produit de facteurs premiers

On décompose $b$ en un produit de facteurs premiers

On regarde les facteurs premiers communs en tenant compte de leurs exposants

\subsubsection{Algorithme a}

$\PGCD(a,b) = \PGCD(a - b, b)$ ou en pseudo code :

\begin{verbatim}
a <- 0
While (a >= b) do:
Begin
    a <- a - b
    q <- q + 1
End
    q <- q
    r <- a

Return PGCD(b,r)
\end{verbatim}

\subsubsection{Algorithme b}

$\PGCD(a,b) = \PGCD(b, a \mod b, a)$ ou en pseudo code:

\begin{verbatim}
R <- 0
A <- a
B <- b

While (B != 0) do:
Begin
    R <- A mod B
    A <- B
    B <- R
End

Return A
\end{verbatim}

\subsection{Théorème de Bézout}

Soit $a,b \in \N$ (vrai aussi pour $\Z$)

Soit $d = \PGCD(a,b)$

Alors $\exists u$ et $v$ dans $\Z$ tq :

$u.a+v.b=d$

\subsubsection{Algorithme de Bézout}

\begin{verbatim}
A  <- a
B  <- b
R  <- 0
U  <- 1
V  <- 0
S  <- 0
T  <- 1
Q  <- 0
CS <- 0
CT <- 1

While (B != 0) do:
Begin
     R  <- A mod B
     Q  <- A div B
     CS <- S
     CT <- T

     A  <- B
     S  <- U - Q.S
     U  <- CS

     B  <- R
     T  <- V - Q.T
     V  <- CT
End

Return U, V
\end{verbatim}

\subsubsection{Nombres premiers entre eux}

On a que $a$ et $b$ sont premiers entre eux ssi $\exists u,v \in \Z$ tq $u.a + v.b = 1$

\subsubsection{Corollaire du théorème de Bézout}

$\grp{a,b}_{\Z} = \Z.\PGCD(a,b) = \PGCD(a,b).\Z$

\subsection{Sous groupes de $\Z$}

On a que si $n \in \N$ alors $\grp{n \Z, +, 0} < \grp{\Z,+,0}$

Sous groupe $0.\Z = 0$ (sous groupe trivial)

Les $n \Z$ tel que $n \in \N$ sont les seuls sous groupes de $\grp{\Z,+,0}$

\subsection{Anneau commutatif unitaire}

On a que $\grp{A,+,.}$ est un anneau commutatif unitaire (souvent appelé anneau) ssi $\grp{A,+,0}$ est un groupe, $\grp{A,.,1}$ est un monoïde et on a la distributivité càd $\forall x,y,z \in A, x(y+z) = xy + xz$

\subsection{Index}

Soit $\grp{G,.,1}$ un groupe et $H < G$ on regardes toutes les classes latérales (COSETS) à gauche de $H$ c'est à dire les parties de $G$ de la forme $gH$

On a que les $gH$ forment une partition de $G$ et il y a une bijection entre $gH$ et $g'H$ donnée par $gh \ap g'h$

Donc si $H$ fini alors tous les $gH$ ont pour cardinal $\vert 1.H \vert = \vert H \vert$ et si $\vert G \vert < \infty$ alors le nombre de classes distinctes de type $gH$ est $\frac{\vert G \vert}{\vert H \vert}$ 

On appel $\frac{\vert G \vert}{\vert H \vert}$ l'index (appelé pareillement dans le cas $\infty$ de $H$ dans $G$)

\subsection{Morphisme de $\Z$ vers groupe cyclique}

Soit $\grp{G,.,1}$ un groupe et $g \in G$

On a un morphisme $\sigma_g : \Z \ap \grp{g} : z \ap g^z$

%% TODO: Nom de ce morphisme ? Est ce un épimorphisme ?

\subsection{Ordre d'un sous groupe engendré}

Soit $\grp{G,.,1}$

Soit $g_1,g_2 \in G$ tq $g_1.g_2 = g_2.g_1$

On a $\ordre(g_1.g_2) \leq PPCM(\ordre(g_1) , \ordre(g_2))$

On a $\PPCM(\ordre(g_1) , \ordre(g_2)) \in \Ker \sigma_{g_1.g_2}$

D'autre part $\Ker \sigma_{g_1.g_2} = \alpha \Z$ avec $\alpha = \ordre(g_1.g_2)$ le générateur

On a $\PPCM(\ordre(g_1) , \ordre(g_2))$ est un multiple de $\ordre(g_1.g_2)$

\subsection{Théorème}

Soit $g \in G$ d'ordre $m$

Soit $g^i$ avec $i \in \tset{0,1,\cdots,m-i}$

Alors $\ordre(g^i) = \frac{m}{\PGCD(m,i)}$

%%TODO: racine primitive mième 25/02

%%TODO: indicateur d'Euler

\subsection{Théorème}

Si $G$ est un groupe d'ordre premier càd $\card{G} = p$ avec $p$ premier alors $G$ est cyclique

\subsection{Petit théorème de Fermat}

Soit $a \in \Z$, soit $p$ premier alors $a^p \equiv a \mod p$

\subsection{Lemme}

Dans un groupe fini $G$ avec $\card{G}=n$ alors $\forall g \in G, g^n = 1$


\section{Groupes de permutations}

\subsection{Permutation}

Soit $X$ un ensemble. Une permutation de $X$ est une bijection $X \ap X$

\subsection{Ensemble des permutations de $X$}

Soit $X$ un ensemble. L'ensemble des permutations de $X$ est noté $S_X$

\subsection{Groupe de permutation}

Soit $X$ un ensemble.

On a que $\grp{S_X,\circ,\mathbb{I}}$ est un groupe de permutation

Avec l'opération de composition de fonctions

Avec le neutre la bijection qui $\forall x \in X, \mathbb{I} : x \mapsto x$

Avec l'inverse la réciproque

\subsection{Ensemble des applications linéaires bijectives}

Soit $V$ un $K$-espace vectoriel de dimension $n$

On note l'ensemble des applications linéaires bijectives $V \ap V$ par ${\mbox{Lin}}_{\mbox{bij}} (V,V)$
 
On a ${\mbox{Lin}}_{\mbox{bij}} (V,V) \cong {GL}_n (K)$

On a ${\mbox{Lin}}_{\mbox{bij}} (V,V) < S_V$

\subsection{Rotation de $\R^2$}

On note l'ensemble des rotations dans $\R^2$ par $\mbox{Rot}(\R^2) = {\mbox{SO}}_2(\R)$

On a ${\mbox{SO}}_2(\R) < S_{\C}$

\subsection{Trouver une permutation dans un groupe}

Soit $\grp{G,.,1}$ un groupe

Pour trouver une permutation dans ce groupe on fixe un $g \in G$

Puis on regarde la bijection $m_g : G \ap G : h \mapsto m_g(h) = g.h$

\subsection{Théorème}

$\tau : \grp{G,.,1} \ap \grp{S_G, \circ, \I} : g \mapsto m_g$ est un monomorphisme de groupe

On a $\im \tau < S_G$ 

On a $G \cong \im \tau$ càd que $G$ plonge dans $S_G$ ce qu'on écrit $G \searrow S_G$

\subsubsection{Remarque}

Tout groupe peut être vu comme un sous groupe de permutation

\subsection{Automorphisme}

On appelle automorphisme de $G$ dans $G$ un isomorphisme $G \ap G$

\subsection{Automorphisme intérieur}

Soit $g \in G$

On considère l'application $\alpha_g$ de $G \ap G$ définie par la conjugaison par $g$

Càd $\alpha_g : G \ap G : h \mapsto g^{-1}.h.g$

Cette application est appelé un automorphisme intérieur\footnote{Inner en anglais.}

On dit intérieur car cet automorphisme est déterminé par $g$

\subsection{Inner}

On appelle Inner l'ensemble des automorphismes intérieur dans $G$

On note $\Inn G = \tset{\alpha_g \vert g \in G}$

\subsection{Théorème}

\begin{itemize}
\item{$\Inn G < S_G$}
\item{$\alpha : G \ap \Inn G : g \mapsto \alpha_g$ est un épimorphisme de groupe}
\item{$\Ker \alpha = \mathcal{Z}(G)$}
\item{$\Inn G \cong G/\Ker \alpha$}
\item{$\Inn G \cong G / \mathcal{Z}(G) \searrow S_G$}	
\end{itemize}

\subsection{Théorème}

On a $\alpha$ est un morphisme de groupes dont $\Ker \alpha = Z(G)$

\subsection{Corollaire}

Par le théorème fondamental on a $G/\Ker \alpha \cong \im \alpha$ càd $G/Z(G) \cong \Inn G$

\subsection{Conjugué}

Soit $h \in G, g \in G$ on dit que $ghg^{-1}$ est le conjugué de $h$ par $g$ ce qu'on note parfois $h^g$.

Remarque : $h^g = h \ioi gh = hg$ 

Remarque : $\mathcal{Z}(G) = \tset{g \in G \vert \forall h \in G, h^g = h}$

\subsection{Commutateur}

%% Je me suis pas trompé ici ?

Soit $h, g \in G$ on note $ghg^{-1} h^{-1}$ par $[g,h]$ et on appelle cet élément de $G$ le commutateur de $g$ et $h$

Remarque : $[h,g] = hgh^{-1}g^{-1} = [g,h]^{-1}$

Remarque : $[g,h] = 1 \ioi h^g = h$

Remarque : $\mathcal{Z}(G) = \tset{g \in G \vert \forall h \in G, [g,h] = 1}$

\subsection{Lemme}

On a $H \normal G \ioi \forall g \in G, \forall h \in H, ghg^{-1} \in H$ càd les conjugués d'éléments de $H$ sont dans $H$

On note $[G,H] = \tset{ghg^{-1}h^{-1} \vert g \in G, h \in H}$

\subsection{Lemme}

La relation $R$ définie par $hRh' \ioi \exists g \in G, h' = ghg^{-1}$ est une relation d'équivalence

\subsection{Lemme}

Si $\card{X} = \card{Y}$ alors $S_X \cong S_Y$

Donc on a $S_X \cong S_n$ avec $\card{X} = n$ 

Avec comme isomophisme $\sigma : S_X \ap S_n : p \mapsto f \circ p \circ f^{-1}$

Où $f : X \ap \N : x_i \mapsto i$

\subsection{Notation du graphe de $p$}

On note la permutation $p \in S_n$ 

$$
p \equiv \begin{pmatrix}
1 & 2 & \dots & n \\
p(1) & p(2) & \dots & p(n)
\end{pmatrix} 
$$

\subsubsection{Exemple de notation}

Prenons par exemple la permuation $p \in S_3$ tq $p = \begin{pmatrix} 1 & 2 & 3 \\ 2 & 3 & 1 \end{pmatrix}$

On note aussi $p \equiv 1 \ap 2 \ap 3 \ap 1$ dans le cas d'un cycle (voir plus loin)

Ce qu'on note aussi $p \equiv (2~3~1)$ qui est la notation classique d'un cycle (voir plus loin)

\subsection{Domaine de la permutation}

%% FIXME: Est-ce que ça s'appelle bien comme ça ?

Soit $p \in S_n$

On note $\dom p = \tset{1, 2, \dots, n} = E_n$

\subsection{Point fixe}

Soit $p \in S_n$

On dit que $i \in E_n$ est un point fixe de $p$ ssi $p(i) = i$

%% TODO: rajouter la notation avec la boucle

\subsection{Ensemble des points fixes}

Soit $p \in S_n$ 

On note l'ensemble des points fixes de $p$ par 

$\Fix p = \tset{i \in E_n \vert p(i) = i}$

\subsection{Support}

Soit $p \in S_n$

On note le support de $p$ par 

$\Supp p = E_n \but \Fix p$

\subsection{Cycle}

Soit $p \in S_n$ 

On dit que $p$ est un cycle de longeur $k \leq n$ s'il existe $j_1, \dots, j_n \in E_n$ tq $\Supp (p) = \tset{j_1, \dots , j_k}$ 

Avec $p \equiv j_1 \ap j_2 \ap \dots \ap j_k \ap j_1 = (j_1~j_2~\dots~j_k)$

Càd $\forall l \in \tset{1, \dots, k}$ on a $p(j_l) = j_{l+1}$ si $l < k$ et $p(j_l) = j_1$ sinon

\subsection{Ordre d'un cycle}

%%FIXME: je suis pas certain que ce soit k l'ordre

L'ordre d'un cycle de longueur $k$ est $k$ càd $p^k = 1$

\subsection{Permutations à supports disjoints}

Soit $p,q \in S_n$ deux permutations

On dit qu'elles sont à support disjoints si $\Supp(p) \cap \Supp(q) = \emptyset$

\subsection{Lemme}

Soit $p \in S_n$ et $i \in \Supp(p)$ alors $p(i) \in \Supp(p)$

\subsection{Lemme}

Si $p,q$ sont à supports disjoints alors $pq = qp$

\subsection{Corollaire}

Si $c_1$ et $c_2$ sont deux cycles à supports disjoints alors :

$\ordre(c_1, c_2) = \PPCM( \ordre(c_1) , \ordre(c_2) )$

\subsection{Lemme}

Toute permutation $p \in S_n$ est un produit (composition) de cycles à supports disjoints

\subsubsection{Exemple et notation}

Prenons par exemple la permutation $p \in S_7$ tq $
p \equiv \begin{pmatrix}
1 & 2 & 3 & 4 & 5 & 6 & 7 \\
1 & 2 & 5 & 6 & 4 & 3 & 7
\end{pmatrix}$

On peut noter $p \equiv 3 \ap 5 \ap 4 \ap 6 \ap 3 \equiv (3~5~4~6)$ en laissant les points fixes

Prenons par exemple la permutation $p \in S_{10}$ tq $
p \equiv \begin{pmatrix} 
1 & 2 & 3 & 4 & 5 & 6 & 7 & 8 & 9 & 10 \\
3 & 6 & 5 & 2 & 1 & 7 & 4 & 8 & 10 & 9
\end{pmatrix}$

On peut noter $p \equiv (1~3~5).(2~6~7~4).(8).(9~10)$ càd un produit de cycles disjoints (car les cycles disjoints commutent)

On enlève généralement les points fixes on a donc $p \equiv (1~3~5).(2~6~7~4).(9~10)$

\subsection{Lemme}

On a $\Supp(p^l) \subset \Supp(p)$

% FIXME : Sure that f:G->H ???
\subsection{Pigeon Hole Principle}

Soit $f : G \ap H$ tq $\card{ G } = \card{ H } < \infty$

Alors on a les équivalences suivantes :

\begin{itemize}
	\item{$f$ est injective}
	\item{$f$ est surjective}
	\item{$f$ est bijective}	
\end{itemize}

\subsection{Transposition}

Un cycle de longueur $2$ est appelé une transposition qu'on note avec la notation de cycle par $(i~j)$ avec $i \neq j$

\subsection{Involution}

Une involution est est une opération qui est son propre inverse.

Dans le cas de permutations, il s'agit des permutations d'ordre 2.

\subsection{Théorème}

\begin{itemize}
	\item{Toute permutation est un produit de transposition}
	\item{$\forall p \in S_n$ la parité $\epsilon(p)$ du nombre de transposition pour écrire $p$ est fixe càd indépendante de la décomposition}
	\item{L'application $\epsilon : S_n \ap \grp{\Z / 2 \Z, + , 0} : p \mapsto \epsilon(p)$ est un épimorphisme}
\end{itemize}

\subsection{Théorème}

Toute permutation $p \in S_n$ est un produit de transpositions

\subsection{Proposition}

Il y a un seul groupe d'ordre 2 à isomorphisme près

\subsection{Proposition}

Il y a seulement 2 groupes d'ordre 4 différents à isomorphisme près

\subsection{Signature}

On note le signe ou la signature d'une permutation $p$ par $\Sgn(p) = (-1)^{\epsilon(p)}$

On a $\Sgn(p) \in \tset{-1,1}$

\subsection{Groupe alterné}

On appelle $\Ker \epsilon = \tset{ p \in S_n \vert \epsilon(p)=0}$ le groupe alterné d'ordre $n$ noté $A_n$

On a donc $\grp{A_n,\circ,Id}$ est un groupe

\subsubsection{Permutation paire}

Une permutation \textit{appartenant} à $A_n$ est une permutation paire

\subsubsection{Permutation impaire}

Une permutation n'appartenant \textit{pas} à $A_n$ est une permutation impaire

\subsubsection{Remarque}

On a $A_n \normal S_n$ clos par conjugaison

\subsection{Remarque}

Soit $H < G$ et $(G:H)=2$ (l'indice de $H$ dans $G$ est $2$ càd il existe seulement deux classes latérales (à gauche ou à droite dans $G$)

Autrement dit $\vert G \vert = 2 \vert H \vert$ alors $H \normal G$

Ce qui est faux pour l'indice 3

\subsection{Groupe simple ou premier}

Soit $G$ un groupe, on dit que $<G,.,1>$ est simple ou premier si $G$ n'a pas de sous groupes normal autre que $G$ et $\tset{1}$

\subsubsection{Théorème}

On a $A_n$ est simple si $n \geq 5$

\subsubsection{Proposition}

On a que si $G$ est simple et $G$ est cyclique fini alors $\vert G \vert$ est premier

On a donc qu'un tel $G$ est un $\grp{\Z/p \Z, + ,0}$ avec $p$ premier

Et donc par Lagrange que ses sous-groupes sont d'ordre $1$ ou $p$ càd les seuls sous-groupes sont $\tset{0}$ et $\Z / p \Z$

Donc les $\Z / p \Z$ sont simples

\subsection{Groupe dérivé}

Soit $G$ un groupe, on appelle le groupe dérivé $G' = \grp{\tset{[a,b] \vert a,b \in G}}$ càd le sous-groupe engendré par l'ensemble des commutateurs

\end{document}
