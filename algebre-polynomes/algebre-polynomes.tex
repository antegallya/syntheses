\documentclass[a4paper,10pt]{article}

\usepackage{amsfonts}
\usepackage{amsmath}
\usepackage{amssymb}
\usepackage{amsthm}
\usepackage[utf8]{inputenc}
\usepackage[francais,french,frenchb]{babel}
\usepackage[T1]{fontenc}
\usepackage{fullpage}
\usepackage{graphicx}
\usepackage[perpage]{footmisc}
\usepackage{srctex}
\usepackage{verbatim}

\author{Pierre Hauweele}
\title{Synthèse Algèbre I - Polynômes - 2007-2008}
\date{}

\newcommand{\ap}{ \rightarrow} % Application
\newcommand{\mt}[1]{\widetilde{ #1 }} % Wide tilde in math mode
\newcommand{\grp}[1]{\left\langle #1 \right\rangle} % Group
\newcommand{\set}[1]{\left\lbrace #1 \right\rbrace } % Set
\newcommand{\im}{\mathrm{Im}\:} % Im(age)
% Underbrace with argument
\newcommand{\underb}[2]{\underset{ #1 }{\underbrace{ #2 }}}
\newcommand{\st}[1]{#1^{\star}} % .^star
\newcommand{\so}{\Rightarrow} % so/implies
\newcommand{\ioi}{\Leftrightarrow} % If and Only If
\newcommand{\ZZ}{\mathbb{Z}} % Integer set
\newcommand{\RR}{\mathbb{R}} % Real set
\newcommand{\NN}{\mathbb{N}} % Natural set
\newcommand{\QQ}{\mathbb{Q}} % Rational set
\newcommand{\CC}{\mathbb{C}} % Complex set
\newcommand{\pgcd}{\mathrm{pgcd}} % french g.c.d=p.g.c.d
\newcommand{\ppcm}{\mathrm{ppcm}}
\newcommand{\id}{\mathrm{Id}} % Identity
\newcommand{\rstrct}[2]{{ #1 }_{\upharpoonright_{ #2 }}} % Operator restriction
\newcommand{\transposee}[1]{{\vphantom{#1}}^{\mathit t}{#1}} % Transposée
\newcommand{\abs}[1]{\left\vert #1 \right\vert} % Absolute
\newcommand{\mfootnote}[1]{\up{(}\footnote{#1}\up{)}}
\begin{document}
\maketitle
\tableofcontents
\newpage
\section{Anneaux, divers}
 \subsection{L'ensemble des inversibles $A^\star$\label{ensinv} (est un groupe)}
  L'ensemble des inversibles d'un anneau $\grp{A,+,.,0,1}$ est appelé $A^\star$. C'est-à-dire :
  $$A^\star=\set{a\in A | \exists a' \in A, a . a' = 1}$$

  $\grp{A^\star,.,1}$ est un groupe.

 \subsection{Théorème de Wedderburn}
  Un corps fini est nécéssairement commutatif.

 \subsection{Ensemble des matrices sur un anneau\label{matsuranneau}}
  Si $\grp{A,+,.,0,1}$ est un anneau, alors $\grp{M_k(A),+,.,0,\id}$ est un anneau.

 \subsection{Fonction déterminant de matrice en tant que morphisme\label{detmorphisme}}
  La fonction $\det(.)$ est un morphisme de groupes multiplicatifs\mfootnote{Ca n'est pas le cas pour les groupes additifs}
 
 \subsection{Ensemble des matrices de déterminant $1$ en valeur absolue : ${SL}_k(B)$}
  On note l'ensemble des matrices avec coefficients dans $B$ dont le déterminant vaut $1$ ou $-1$ :
  $${SL}_k(A)=\set{A \in B ~| \det A|=1}$$

 \subsection{Ensemble des matrices sur les entiers et déterminants}
  $\grp{M_k(\ZZ)^\star,+,.,0,\id}$ est un anneau et de plus :
  $$\grp{M_k(\ZZ)^\star,.,\id} \overset{\det}{\ap}\grp{\set{-1,1},.,1}$$
  En effet, $\grp{\ZZ,+,.,0,1}$ est un anneau donc $\grp{M_k(\ZZ),+,.,0,\id}$ est un anneau (\ref{matsuranneau}). On a que $\grp{M_k(\ZZ)^\star,.,\id}$ est un groupe (\ref{ensinv}).
  Soit $A \in \grp{M_k(\ZZ)^\star,.,\id}$
  \begin{eqnarray*}
   \det(A.A^{-1})=\det(\id) &=& 1\\
   \det(A).\det(A^{-1})&=&1 ~~~~~~\mbox{(Car $\det(.)$ morphisme de groupes)}
  \end{eqnarray*}
  Ce qui implique que $\det(A)$ et $\det(A^{-1})$ sont inverses dans $\ZZ$. Et les seuls inversibles dans $\ZZ$ sont $1$ et $-1$. Donc $\det(A) \in \set{1,-1}$ qui est un groupe multiplicatif.

  \paragraph{Remarque}
  $$M_k(\ZZ)^\star = {SL}_k(\ZZ) = \set{A \in M_k(\ZZ) ~| \abs{\det A} = 1}$$

 \subsection{Morphismes d'anneau}
  Soient $\grp{A,+,.,0,1}$ et $\grp{B,+,.,0,1}$ deux anneaux.
  Une application $\sigma: A \ap B$ est un morphisme d'anneaux ssi :
  \begin{enumerate}
   \item $\sigma$ est un morphisme de groupes de $\grp{A,+,0}\ap\grp{B,+,0}$\\
         c'est-à-dire : $\forall u,v \in A, \sigma(u+v) = \sigma(u)+\sigma(v)$
   \item $\sigma$ est un morphisme de monoïde avec neutre, c'est-à-dire :
         \begin{enumerate}
	  \item $\forall a,a' \in A, \sigma(a,a')= \sigma(a).\sigma(a')$
	  \item $\sigma(1)=1$
	 \end{enumerate}
  \end{enumerate}

 \subsection{Lemme: l'image d'un morphisme d'anneaux est un sous-anneau}
  L'image d'un morphisme d'anneau $A \ap B$ est un sous-anneau de $B$.


 \subsection{Propriétés sur le noyau d'un morphisme d'anneaux}
  Soient $A$ et $B$ deux anneaux et $\sigma: A \ap B$ un morphisme d'anneaux.
  \begin{itemize}
   \item $\ker{\sigma}$ est un idéal bilataire de $A$.
   \item $\ker{\sigma}$ est un sous-groupe additif de $A$\mfootnote{Et non pas un sous-anneau}.
  \end{itemize}

 \subsection{Sous-anneau}
  Une partie $B$ de $\grp{A,+,.,0_A,1_A}$, un anneau,
  est un sous-anneau de $A$ ssi : 
  \begin{enumerate}
   \item $\grp{B, +}$ est un sous-groupe de $\grp{A,+}$
   \item $\forall x,y \in B, x.y \in B$
   \item $1_A \in B$
  \end{enumerate}


  \paragraph{Remarque} Si $I$ est le noyau d'un morphisme d'anneaux, alors ce noyau est soit injectif, soit nilpotent d'ordre 1.

 \subsection{Sélection d'éléments avec des matrices}
  Soit $E_{ij}$ la matrice définie par 
  $$E_{nm}=
  \begin{cases}
   1 \mbox{ si }(n=i) \land (m=j)\\
   0 \mbox{ sinon}
  \end{cases}$$
  Alors, avec $M$ une matrice : $E_{ij}$ est la matrice composée de $0$ sauf sa $i^{ieme}$ ligne qui est composée de la $j^{ieme}$ ligne de $M$.

  La multiplication à gauche par $E_{ij}$ aura le même effet, mais sur les colonnes, on aura la $j^{ieme}$ colonne de $M$ en $i^{ieme}$ colonne de la matrice résultat.

  Ainsi, en combinant une multiplication à droite et à gauche, on peut sélectionner un unique élément.

 \subsection{Permutations sur les matrices}
  Soit $T_{ij}$ la matrice $\id$ dont on permuté la $i^{ieme}$ et la $j^{ieme}$ ligne. Alors $T_{ij}\cdot M$ est la matrice $M$ dont on a effectué la même permutation.

  \paragraph{Remarque} $\det T_{ij} \in \set{-1,1}$ indique la parité de la permutation associée à $T_{ij}$

  \subsection{Théorème fondamental pour les anneaux \label{thmfondamental}}
  Soit $\grp{A,+,.,0,1}\overset{\sigma}{\ap}\grp{B,+,.,0,1}$. Alors
  $$\grp{A,+,.,0,1}/\ker{\sigma} \cong \grp{\im \sigma,+,.,0,1}$$
  L'isomorphisme associé est $\tau (a+\ker{\sigma})\overset{\Delta}{=} \sigma(a)$

 \subsection{Anneau réduisible à un corps si ses inversibles avec le neutre est lui-même}
  Soit $A$ un anneau, si $A^\star=A\setminus \set{0}$ alors $A$ est un corps.

 \subsection{Idéaux}
  \subsubsection{Idéal à gauche/droite/bilataire}
    Une partie $I$ d'un anneau $A$ est un idéal à gauche ssi : 
    \begin{enumerate}
     \item $I$ est un sous-groupe additif de $A$
     \item $\forall a\in A, \forall i \in I, a.i \in I$
       \mfootnote{Pour un idéal à droite on demandera $i.a \in I$}
    \end{enumerate}

    On parle d'idéal bilataire quand il l'est à gauche et à droite.

  \subsubsection{Idéal bilataire de matrices sur les réels}
   Un idéal bilataire $I$ de $M_k(\RR)$ est soit $\set{0}$, soit $M_k(\RR)$.

  \subsubsection{Idéal de $A$ inter ses inversibles ou, l'idéal est l'anneau}
   Soit $I$ idéal de $A$, on a alors : 
   $$I \cap A^\star \neq \emptyset \ioi I=A$$

  \subsubsection{Idéal propre}
   $I$ est un idéal propre de $A$ si et seulement si $I\neq A$

  \subsubsection{Idéal propre maximal}
   \paragraph{1\up{ière} version} $I$ est un idéal propre \underline{maximal} de $A$ si et seulement si $I$ est un idéal propre de $A$ et si $J$ idéal propre de $A$ et $I \subseteq J$, alors $J=I$

    Ou encore, si $J$ idéal de $A$ et $I \subsetneq J$, alors $J=A$

   \paragraph{2\up{ième} version} $A$ anneau commutatif et $I$ idéal propre de $A$, alors $I\neq A$ est un idéal maximal dans $A$ si et seulement si 
   $$\forall a' \in A\setminus I, I+a'A=A$$

  \subsubsection{Idéal maximal et corps}
   Soit $I$ idéal de $A$, on a :
   $$A/I \mbox{ est un corps} \ioi I \mbox{ est maximal}$$

  \subsubsection{Idéal premier}
   $I$ est un idéal premier de $A$ si et seulement si :
    $$\forall a,b \in A, ab\in I \so (a\in I \mbox{ ou } b\in I)$$
  
  \subsubsection{Idéal principal}
   On dit que $I$ est un idéal principal de $K$ anneau si il est généré par un
   seul élément. C'est-à-dire que $I=a.K$ avec $a \in K$\mfootnote{Dans le cours
   on ne parle que d'idéal principal pour les polynômes, c'est-à-dire qu'ils
   sont de la forme $I=gK[X]$.}.

   \paragraph{Remarque} Le notation adoptée ci haut est utilisée dans le cas où
   la multiplication est commutative; quand elle ne l'est pas on parle d'idéal
   principal à droite ou à gauche.

 \subsection{Anneau principal}
  Un anneau est dit principal si il est commutatif et si tous ses sous anneaux
  sont principaux.
  \paragraph{Exemples}
  \begin{enumerate}
    \item Les $K[X]$ avec $K$ corps commutatif.
    \item Les $\ZZ$ car ses idéaux sont de la forme $n\ZZ$.
    \item $K$ un corps commutatif, car $K$ et $\set{0}$ sont les seuls idéaux
      d'un corps. 
  \end{enumerate}

 \subsection{Diviseur de zero}
  \subsubsection{Définition}
   Soit $A$ un anneau, $a \in A$ est dit diviseur de zero à gauche si :
    $$a\neq 0, \exists b\neq 0\in A, ab=0$$
  \subsubsection{Propriétés inversibles}
   Si $a\in A^\star$, alors $a$ n'est pas diviseur de zero.

 \subsection{Anneau intègre}
  Un anneau est dit intègre si il n'a pas de diviseurs de zero et est commutatif.

  \paragraph{Remarque} Un anneau intègre fini est un corps.

 \subsection{Anneau unitaire}
  $A$ un anneau est dit unitaire si et seulement si la loi $\cdot$ possède un neutre $1$.

  \subsection{Quotient d'anneau et d'idéal, intègre, idéal premier
  \label{qoaiiip}}
  Soit $A$ un anneau commutatif, $I$ idéal de $A$, on a :
   $$A/I \mbox{ intègre} \ioi I \mbox{ idéal premier de } A$$

 \subsection{Relation anneaux, idéaux maximal/premier, quotient corps/intègre}
   $$A/I \mbox{ corps} \ioi I \mbox{ maximal} 
   \so I \mbox{ premier} \ioi A/I \mbox{ intègre}$$
  Rappel : si $A/I$ est intègre et fini alors $A/I$ est un corps


\section{Polynômes}
 \subsection{Polynômes : degré et multiplication}
  On définit le degré du polynôme nul : 
   $$\deg 0 = -\infty$$
  On remarque les règles suivantes sur les degrés de polynômes :
   \begin{itemize}
    \item $\deg(p+q) \leq \max\set{\deg p, \deg q}$
    \item $\deg(p\cdot q) = \deg p + \deg q$
   \end{itemize}

  La multiplication de deux polynômes donne ce qu'on appelle la multiplication de Cauchy : 
  $$(p\cdot q)_m = \sum_{i+j=m} p_i \cdot q_j$$

 \subsection{Polynômes sur un anneau commutatif}
  Si $A$ est un anneau commutatif, alors $A[X]$ est un anneau commutatif.
 
 \subsection{Théorème central\label{thmcentral}}
  Dans $K[X]$, $I$ idéal de $K[X]$ est nécéssairement de la forme $qK[X]$
  avec $q \in K[X]$ (et ils sont donc aussi des idéaux principaux). On a alors
  l'équivalence entre les propriétés suivante :
  \begin{itemize}
   \item $qK[X]$ est premier
   \item $qK[X]$ est maximal
   \item $q$ est irréductible dans $K[X]$
  \end{itemize}

 \subsection{Morphisme d'évaluation}
  L'application $eval_a : K[X] \ap L \ni a : p \ap p(a)$ est un morphisme d'anneau.
  
  On a donc bien (ou en effet) : 
  \begin{eqnarray*}
   eval_a(p+q)=&eval_a(p)+eval_a(q)\\
   eval_a(p\cdot q)=&eval_a(p)\cdot eval_a(q)
   \end{eqnarray*}

 \subsection{Pas de diviseurs de zero pour les polynômes}
  $K[X]$ n'a pas de diviseurs de zero.

 \subsection{Polynôme divisible par un complexe et son conjugué}
   Si un complexe divise un polynôme, son conjugué le divise également.

 \subsection{Propriétés des $n\ZZ$}
  Les $n\ZZ$ sont toujours sous-groupe de $\grp{Z,+}$ et ce sont les seuls.
  
  Les idéaux de $\grp{\ZZ,+,0}$ sont les $\grp{n\ZZ,+,0}$.

  $$\grp{n\ZZ,+} \mbox{ premier} \ioi n \mbox{ premier}$$

  Si $p$ est premier, alors $p\ZZ$ est un idéal \underline{maximal}. 

 \subsection{Polynômes associés}
  On dit que $g$ et $\mt{g} \in K[X]$ sont associés si et seulement si :
  $$\exists v\in (K[X])^\star = K^\star, \mt{g} = ug~ ~(\mbox{ou } g=u^{-1}g)$$

  \paragraph{Relation d'équivalence}
   $R(g,\mt{g}) \ioi g$ est associé à
   $\mt{g}$ est une relation d'équivalence.
 
 \subsection{Critères de (ir)réductibilité}
  \mfootnote{Cette section ne comporte ne contient pas tous les éléments qui
  pourraient servir de critère, par exemple, le théorème central peut servir de
  critère d'irréductibilité mais n'est pas repris ici.}
 
  \subsubsection{Irréductibilité en termes de diviseurs}
   $g \in K[X]$ avec $\deg g \ge 1$ est irréductible dans $K[X]$ si et seulement
   si les seuls diviseurs de $g$ sont les inversibles dans $K[X]$ et les
   $\mt{g}$ associés à $g$. Ou :
   $$\forall u,v \in K[X], (g=uv) \so (\deg u=0 \lor \deg v=0)$$

  \subsubsection{Polynômes de degré 2 ou 3}
   Un polynôme de degré 2 ou 3 est irréductible dans $K[X]$ si et seulement si
   il n'a pas de racies dans $K$.
  
  \subsubsection{Par degrés de décomposition}
   On peut obtenir des informations sur la réductibilité de polynômes en
   regardant les possibilités de décomposition de ces polynômes. Par exemple,
   pour un polynôme de degré 4, ses décompositions possiblies sont : 
   \begin{enumerate}
     \item 1 1 1 1
     \item 1 2 1
     \item 2 2
     \item 3 1
     \item 4 
   \end{enumerate}
   Si on demande sa réductibilité dans $\CC$ et que l'on sait qu'il n'a pas de
   racines dans $\RR$ (par exemple parcequ'il est toujours positif), la
   décomposition (3 1) devient impossible. En effet, si il existe une
   décomposition en un polynôme de degré 1, alors il s'agit d'une racine
   complexe, mais tout polynôme possédant un racine complexe possède également
   comme racine le conjugué de ce complexe.

  \subsubsection{Théorème de Gauss}
   Soit $g \in \ZZ[X]$, il existe $g_1,g_2 \in \QQ[X]$ tels que $g=g_1\cdot g_2$
   est équivalent à : il existe $\mt{g_1},\mt{g_2} \in \ZZ[X]$ tels que
   $g = \mt{g_1} \cdot \mt{g_2}$
   et $\deg \mt{g_1} = \deg g_1$ et $\deg \mt{g_2} = \deg g_2$
   \paragraph{C'est-à-dire} un polynôme dans $\ZZ[X]$ est réductible dans
   $\QQ[X]$ si et seulement si il est réductible dans $\ZZ[X]$.
  
  \subsubsection{Critère d'Eisenstein ($E_2$)}
   % TODO: détails 14/05/08 p3
   Soit $g=a_n X^n+a_{n-1}X^{n-1}+ \cdots + a_1 X + a_0$ avec les
   $\set{a_0, \hdots, a_n} \in \ZZ$. Si il existe $p$ premier tel que :
   \begin{itemize}
     \item $p$ divise tous les $a_0, \hdots, a_{n-1}$.
     \item $p$ ne divise pas $a_n$.
     \item $p^2$ ne divise pas $a_0$.	
   \end{itemize}
   Alors $g$ est irréductible dans $\ZZ[X]$
   \paragraph{Remarques} 
   \begin{itemize}
     \item Si $g \in \QQ[X]$, on peut appliquer le critère sur son polynôme
       associé en le multipliant par $\alpha \in \ZZ$ assez grand pour éliminer
       les dénominateurs. Le critère donne le même résultats pour les polynômes
       associés.
   \end{itemize}
  \subsubsection{Critère des racines dans $\QQ$}
   Soit $g \in \QQ[X]$ avec $g=g_n X^n + \cdots + g_1 X + g_0$. Soit 
   $\alpha = \frac{a}{b} \in \QQ, a\in \ZZ, b \in \NN_0$ et $\pgcd (a,b)=1$.
   Soit $g=\frac{a_n}{b_n} X^n + \cdots + \frac{a_0}{b_0}$. $\alpha$ est racine
   de $g$ si et seulement si : \\
   $\alpha$ racine de $a_n c_n X^n + \cdots + a_0 c_0 \in \ZZ[X]$ avec
   $c_j=\frac{\ppcm (b_m, \hdots, b_0)}{b_j}$.
  \subsubsection{Lemme : Critère des racines}
   $\frac{a}{b} \in \QQ$ racine de $g=g_n X^n + \cdots + g_0$ si seulement si
   $a/g_0$\mfootnote{``divise''} et $b/g_n$ dans $\ZZ$.
   \paragraph{Exemple} $P=X^3-2X+7$ est-il réductible dans $\QQ[X]$ ? Si oui,
    $\alpha=\frac{a}{b}$ et $\pgcd (a,b)=1$, le lemme nous dit : 
    $$a/ 7 \land b/1$$
    ce qui nous laisse comme possibilités pour $\alpha$ :
    $$\alpha=\frac{1}{1},\frac{-1}{1},\frac{7}{1},\frac{-7}{1}=1,-1,7,-7$$
    mais aucune de ces valeurs n'est racine de $P$, donc il ne possède pas de
    racines dans $\QQ$ et comme il est de degré 3 et n'a pas de racines dans
    $\QQ$, il est irréductible dans $\QQ[X]$.
\section{Exemples de  raisonnements}
  \paragraph{Soit $q$  irreductible dans $K[X]$ et $e$ racine de $q$
  dans $L \supset K$} alors $eval_{e} : K[X] \ap L : p(x)\ap p(e)$
  a un noyau $\ker eval_{e} = qK[X]$. Par (\ref{thmfondamental}) on a 
  $\im eval_{e} \cong K[X]/\ker eval_{e}$ et par le théorème central
  (\ref{thmcentral}), on a que $qK[X]$ est maximal et par (\ref{qoaiiip}) on a
  que $K[X]/qK[X]$ est un corps.

\end{document}
