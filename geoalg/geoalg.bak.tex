\documentclass[reqno,a4paper,10pt]{report}

\usepackage{amsfonts}
\usepackage{amsmath}
\usepackage{amssymb}
\usepackage{amsthm}

\usepackage[utf8]{inputenc}
\usepackage[francais]{babel}
\usepackage{xspace,aeguill,ae}
\usepackage[T1]{fontenc}
\usepackage{fullpage}
\usepackage{verbatim,mathrsfs,enumerate,url}
\usepackage[dvips]{color}
\usepackage{nag}
\usepackage{mathtools} % Xtensible arrows
\usepackage[all]{xy} % Commutative diagrams

\author{Pierre Hauweele \and Noémie Meunier}
\title{Introduction à la géométrie algébrique/algèbre commutative}
\date{2009-2010}
\setcounter{secnumdepth}{3}
\setcounter{tocdepth}{3}

%%%%%%%%%%%%%%%%%%%%%%%%%%%%%%%%%%%%%%%%%%%%%%%%%%%%%%%%%%%%%%%%%%%%%%%%
\newcommand{\mt}[1]{\widetilde{#1}} % Wide tilde in math mode
\newcommand{\gen}[1]{\left\langle #1 \right\rangle} % Generator
\newcommand{\set}[1]{\left\lbrace #1 \right\rbrace} % Set
\newcommand{\im}{\mathrm{Im}\:} % Im(age)
% Underbrace with argument
\newcommand{\underb}[2]{\underset{#1}{\underbrace{#2}}}
\newcommand{\st}[1]{#1^{\star}} % .^star
\newcommand{\IZ}{\ensuremath{\mathbb{Z}}\xspace} % Integer set
\newcommand{\IR}{\ensuremath{\mathbb{R}}\xspace} % Real set
\newcommand{\IN}{\ensuremath{\mathbb{N}}\xspace} % Natural set
\newcommand{\IQ}{\ensuremath{\mathbb{Q}}\xspace} % Rational set
\newcommand{\IC}{\ensuremath{\mathbb{C}}\xspace} % Complex set
\newcommand{\so}{\Rightarrow}
\newcommand{\soo}{\Longrightarrow}
\newcommand{\inj}{\hookrightarrow}
\newcommand{\surj}{\twoheadrightarrow}
\newcommand{\bij}{\overset{\!\sim}{\to}} % Bijection
\newcommand{\pgcd}{\mathrm{pgcd}} % french g.c.d=p.g.c.d
\newcommand{\ppcm}{\mathrm{ppcm}}
\newcommand{\id}{\ensuremath{\mathrm{Id}}} % Identity
\newcommand{\rstrct}[2]{{#1}_{\upharpoonright_{#2}}} % Operator restriction 
\newcommand{\transpose}[1]{{\vphantom{#1}}^{\mathit t}{#1}} % Transpose
\newcommand{\abs}[1]{\left\vert #1 \right\vert} % Absolute
\newcommand{\adh}[1]{\mathrm{adh}\left( #1\right)}
\newcommand{\overcircle}[1]{\stackrel{\ \circ}{#1}}
\newcommand{\ord}{\mathrm{ord}} % order of an element
\newcommand{\pfootnote}[1]{\up{(}\footnote{#1}\up{)}} % Footnote w/ (.)
\newcommand{\Homgrp}[1]{\text{Hom}_{\text{grp}}\left(#1\right)}
\newcommand{\Autgrp}[1]{\text{Aut}_{\text{grp}}\left(#1\right)}
\newcommand{\Autens}[1]{\text{Aut}_{\text{ens}}\left(#1\right)}
\newcommand{\Autann}[1]{\text{Aut}_{\text{ann}}\left(#1\right)}
\newcommand{\Aut}[1]{\text{Aut}\left(#1\right)}
\newcommand{\Int}[1]{\text{Int}\left(#1\right)}
\newcommand{\Endgrp}[1]{\text{End}_{\text{grp}}\left(#1\right)}
\newcommand{\Endom}[2]{\text{End}_{#1}\left(#2\right)}
\newcommand{\Orb}[2]{\text{Orb}_{#1}\left( #2 \right)}
\newcommand{\Stab}[2]{\text{Stab}_{#1}\left( #2 \right)}
\newcommand{\Card}[1]{\text{Card}\left( #1 \right)}
\newcommand{\Nmid}{\not\Big| \ } % divides not
\newcommand{\Such}{\ \Big| \ }
\newcommand{\such}{\ | \ }
\newcommand{\ioi}{\Leftrightarrow} % If and Only If (tiny \iff)
\newcommand{\gengrp}[1]{\gen{#1}_{\text{grp}}} % Gen as group
\newcommand{\genann}[1]{\gen{#1}_{\text{ann}}} % Gen as ring (fr anneau)
\newcommand{\tdef}[1]{\underset{\text{def}}{ #1 }}
\newcommand{\tnot}[1]{\overset{\text{not}}{ #1 }}
\newcommand{\ev}[1]{\mathrm{ev}_{ #1 }}

\newtheorem*{theorem*}{Théorème}
\newtheorem*{lemma*}{Lemme}
\newtheorem*{proposition*}{Proposition}

\makeatletter
\AtBeginDocument{%
  \let\olditemize=\itemize%
  \renewenvironment{itemize}{%
    \olditemize%
  }{%
    \@noparlisttrue%
    \endlist%
  }%
  \let\oldenumerate=\enumerate%
    \renewenvironment{enumerate}{%
    \oldenumerate%
  }{%
    \@noparlisttrue%
    \endlist%
  }%
}%
\makeatother
%%%%%%%%%%%%%%%%%%%%%%%%%%%%%%%%%%%%%%%%%%%%%%%%%%%%%%%%%%%%%%%%%%%%%%%

\begin{document}
\maketitle
\tableofcontents
\chapter{Anneaux}
\section{Anneaux}
\subsection{Définitions}
\subsubsection{Groupe (rappel)} 
Un groupe est un ensemble G muni d'une application
$\begin{array}[t]{r@{\ }l}
  G &\to G\\
  (x,y) &\mapsto xy
\end{array}$
tq $\forall x,y,z \in G$
\begin{enumerate}[(1)]
  \item (Associativité) $x.(y.z)=(x.y).z$
  \item (Neutre) $\exists 1_R \in R$ tq $\forall x \in R, 1_R.x = x.1_R = x$
  \item (Inverse) $\forall x \in G$ , $\exists x^*\in G$, $x.x^* = x^*.x = 1_R$
\end{enumerate}

\subsubsection{Anneau}
Un anneau $R$ est un groupe abélien $(R, +)$ muni d'une application
$\begin{array}[t]{r@{\ }l}
  \mu_R: R \times R &\to R\\
  (x,y) &\mapsto x.y = xy
\end{array}$
telle que $\forall x,y,z \in R$ on a :
\begin{enumerate}[(1)]
  \item (Associativité) $x.(y.z)=(x.y).z$
  \item (Neutre) $\exists 1_R \in R$ tq $\forall x \in R, 1_R.x = x.1_R = x$
  \item (Distributivité) 
    $\begin{array}[t]{r@{\ }l}
      x.(y+z) &= x.y + x.z\\
      \text{et }~ ~ (y+z).x &= y.x + z.x
    \end{array}$
\end{enumerate}

\paragraph{Remarques}
(1): permet d'écrire $xyz$ et $x^n, n \in \IN$.\\
(2): $1_R$ est unique.\\
(3) équivaut à $\mu_R$ est un bimorphisme de groupe, i.e.
$\forall x \in R$,
$\left.\begin{array}{r@{\ }l}
  R &\to R\\
  y &\mapsto xy
\end{array}\right\}$
et
$\left.\begin{array}{r@{\ }l}
  R &\to R\\
  y &\mapsto yx
\end{array}\right\}$
morphismes de $(R,+)$\\
En particulier $\forall x \in R, x.0_R=0_R.x=0_R$ (0 est absorbant).

\vskip 1em
Il y a $3$ éléments remarquables dans $R$ : $0_R$, $1_R$ et
$-1_R$.

\paragraph{Remarque}
$0_R = 1_R \equiv \forall x \in R, x = 0_R \iff R = 0_R$: anneau
$\left\{\begin{array}{l}
  \text{nul}\\
  \text{trivial}
\end{array}\right.$

\paragraph{Définition: $R$ anneau commutatif} si de plus $\forall x,y \in R,
x.y = y.x$ (la 2\up{e} loi est commutative)\\
Exemples :
% Commentaire : tu voulais faire qqch de spécial avec
% l'accolade dans : 
%  \item{Anneaux de nombres:} ils sont commutatifs
% ?
% Et quoi les applications à valeurs dans un anneau ?
\begin{enumerate}[1.]
  \item{Anneaux de nombres:} ils sont commutatifs
  \item{Anneaux de tranformations:} non commutatifs en général
  \item{Applications à valeurs dans un anneau $(R,+,.)$}
\end{enumerate}

\subsubsection{$R^\times$, ensemble des inversibles}
L'ensemble des inversibles de $R$ est noté $R^\times$ et est un groupe
multiplicatif (commutatif si $R$ l'est) :
\[R^\times = \set{x \in R \such \exists y \in R, x.y =y.x = 1_R}\]

\subsubsection{Corps}
Un corps est un anneau commutatif K tel que $K^{\times} = K\setminus\set{0}$.

\subsubsection{Entiers de Gauss}
L'anneau des entiers de Gauss est $\IZ[i]=\set{a+ib; a, b \in IZ}$, muni
de $+$ et $.$

\subsection{Lemme: Centre de ${\IZ[i]}^\times$}
\[{\IZ[i]}^\times = \set{1,-1,i,-i} = \gen{i}_{\IC^\times} \simeq \IZ/4\IZ\]
 \begin{comment}
  preuve 17/09/09 p1
\end{comment}


\section{Sous-anneaux}
\subsection{Définition : Sous-anneau}
Soit $R=(R,+,.)$ un anneau et $S \subseteq R$. $S$ est un sous-anneau de $R$
si :
\begin{itemize}
  \item $(S,+)$ est un sous-groupe de $(R,+)$
  \item $1_R \in S$
  \item $\forall x,y \in S, xy \in S$
\end{itemize}

\paragraph{Remarque} $(S^\times, .)$ sous-groupe de $(R^\times, .)$.

\subsection{Chaînes de sous-anneaux}
\begin{itemize}
  \item $\IZ \subseteq \IQ \subseteq \IR \subseteq \IC$ chaine de
    sous-anneaux.
  \item $\IZ \subseteq \IZ[i] \subseteq \IC$ chaîne de sous-anneaux.
  \item $\IZ^\times < {\IZ[i]}^\times < \IC^\times$ chaîne de sous-groupes.
  \item $\IZ + 2i\IZ \tdef{=} \set{a+i2b; a,b \in \IZ}$ est un sous-anneau
    de $\IZ[i]$ et ${(\IZ+2i\IZ)}^\times = \set{-1, 1} = \IZ^\times$.
\end{itemize}

\subsection{Sous-anneaux d'endomorphismes}
On obtient des sous-anneaux des anneaux du type $R=\Endgrp{A}$, où $(A,+)$
groupe abélien, en ajoutant des structures supplémentaires sur $R$ qui sont
compatibles avec la loi $+$.

\paragraph{Exemple} $(A,+) = (\IR^n, +)$
\begin{itemize}
  \item $R = \Endgrp{\IR^n}$
  \item $S=\Endom{\underset{\text{cont}}{\text{grp}}}{\IR^n}$ sous-anneau de
    $R$.
  \item $T=\Endom{\IR-\text{lin}}{\IR^n}=M_n(R)$ sous-anneau de $S$. Et en fait
    $S=T$.
\end{itemize}

De façon similaire, 
\begin{itemize}
  \item $R=\mathcal{A}(\IC, \IC)$.
  \item $S=\mathcal{A}_\text{cont}(\IC, \IC)$ sous-anneau de $R$.
  \item $T=\mathscr{H}(\IC)=\set{\text{applications holomorphes}}$ sous-anneau
    de $S$.
\end{itemize}

\subsection{Centre d'un anneau (est un sous-anneau)}
Le centre de $R$ est :
\[Z(R)=\set{x \in R \such \forall y \in R, xy=yx}\]
$Z(R)$ est un sous-anneau de $R$.

Exemple: $K$ corps, $R=M_n(K)$ anneau. On a $Z(R)=\set{\lambda . \id_n}$

\subsection{Génération d'anneaux}
$T$ anneau, $R,S$ sous-anneaux de $T$.
\begin{itemize}
  \item Alors $R \cap S$ est un sous-anneau de $T$, de $T$ et de $S$.
    On a : $R \cap S =$ le plus grand (pour $\subseteq$) sous-anneau de $T$
    contenu dans $S$ et dans $R$.\\
    Plus généralement, une intersection de sous-anneaux est un sous-anneaux.
  \i tem $P$ une partie de $T$.
    \begin{eqnarray*}
      \gen{P} &\tdef{=}&\bigcap_{
      \left(\begin{array}{c}
        U\text{ sous-anneau}\\
        \text{de } T\\
        \text{tq }P \subseteq U
      \end{array}\right)} U\\
      &=&\text{le plus petit sous-anneau de } T \text{ contenant} P
    \end{eqnarray*}
\end{itemize}

\section{Anneau produit}
Soient $R,S$ anneaux. Alors l'ensemble $R\times S$ est muni d'une structure
d'anneau par :
\begin{itemize}
  \item $(x_1, y_1)+(x_2, y_2) = (x_1+x_2, y_1+y_2)$
  \item $(x_1, y_1).(x_2, y_2) = (x_1x_2, y_1y_2)$
\end{itemize}
On a : $0_{R\times S}=(0_R, 0_S)$ et $1_{R\times S}=(1_R, 1_S)$.\\
De plus, $(R\times S)^\times = R^\times\times S^\times$.
\paragraph{N.B :}
Les sous-ensembles de $R\times S$, $\set{(x,0_S); x \in \IR}$ et $\set{(0_R,
y); y \in S}$ sont des sous-groupes de $(R\times S, +)$ mais ne sont pas
des sous-anneaux (si $S \neq \set{0}$, respectivement $R \neq \set{0}$).

\paragraph{Plus généralement}
on construit un produit quelconque d'anneaux. Par exemple : \\
$R$ anneau. $\mathscr{S}(R) := \set{(u_n)_{n\in \IN}; u_n \in \IR}$.\\
$\mathscr{S}(\IQ)=\prod_{n\in \IN} \IQ$\\
$\mathscr{C}(\IQ)=\set{(u_n)_{n\in \IN} \in \mathscr{\IQ} \such (u_n)_{n\in
\IN} \text{ de Cauchy}}$ (sous anneau de $\mathscr{S}$)

\section{Morphismes}
\subsection{Définition : morphisme d'anneaux}
Soient $R,S$ anneaux\\
Un morphisme d'anneaux $f$ de $R$ dans $S$ est une application $f: R \to S$
telle que $\forall x,y \in R$ :
\begin{enumerate}[(i)]
  \item $f(x+y) = f(x)+f(y)$
  \item $f(x.y) = f(x).f(y)$
  \item $f(1_R) = 1_S$
\end{enumerate}
                 
\paragraph{Remarques:}
\begin{itemize}
  \item (i) $\iff f$ morphisme de groupes $(R,+) \to (S,+)$; on a alors
    $f(0_R) = 0_S$ et $f(-x) = -f(x)$\\
    (ii) et (iii) : compatibilité avec les 2\iemes lois
  \item Si $S$ anneau non nul, l'application
    $\left\{
    \begin{array}{r@{\ }l}
      R &\to S\\
      x &\mapsto 0_S
    \end{array}\right.$ 
    est un morphisme de groupes, mais pas d'anneaux.
  \item Si $f: R \bij S$ morphisme d'anneaux bijectif, alors $f^{-1}: S \bij R$
    morphisme d'anneaux.
\end{itemize}     

\paragraph{Rmq:}
$R \simeq S \tdef{\iff} \exists$ isomorphisme d'anneau $R \bij S$
est une relation d'équivalence.

\paragraph{Rmq:}
Soit $R$ anneau non nul. Il existe un unique morphisme d'anneau
\[\begin{array}{r@{\ }l}
  \IZ &\to R\\
  1 &\mapsto 1_R\\
  n &\mapsto n.1_R
\end{array}\]

%TODO: Exemples ?

\subsection{Ker et Im}
Soient $R$, $S$ anneaux et $f: R \to S$ morphisme d'anneau. En particulier,
$f: (R,+) \to (S,+)$ morphisme de groupe, donc $\ker f = \set{x \in R
\such f(x) = 0_S} \text{ sous-groupe de } (R,+)$\\
et $\im f = \set{f(x) ; x \in R} \text{ sous-groupe de } (S,+)$.

On a : $
\begin{array}[t]{r@{\ }l}
  f \text{ surjective} &\iff \im f = S\\
  f \text{ injective} &\iff \ker f = \set{0_R}
\end{array}$

\subsubsection{$\im f$ est un sous-anneau}
$\im f$ est un sous anneau de $S$.

\subsubsection{$\ker f$ est un ideal bilatère}
$S$ anneau non nul $\so \ker f$ n'est pas un sous-anneau de $R$.\\
Par contre, on a : $\forall y,z \in R, \forall x \in \ker f, xyz \in \ker
f$.\\
C'est à dire $\ker f$ est un idéal bilatère de $R$ ($\so \ker f$ ne
contient pas d'inversibles).

\subsection{Propriétés}
\subsubsection{Image d'un morphisme définie par l'image sur une partie
génératrice}
Soient $R$ anneau, $P\subseteq R$ sous-ensemble tel que $R=\genann{P}$
et $g, f : R \to S$ morphismes d'anneaux.
\[\forall x \in P, f(x)=g(x) \so f=g\]

\subsubsection{Stable par composition}
$R, S, T$ anneaux, $f:R \to S, g: S \to T$ morphismes d'anneau $\so g\circ f$
morphisme d'anneau.

\paragraph{Remarque} On obtient ainsi le groupe $(\Autann{R}, \circ)$. Mais on
n'a pas un anneau.
\begin{comment}
  \pfootnote{détails 01/10/09 p1}
\end{comment}
\begin{itemize}
  \item $\Autann{\IZ} = \set{\id}$
  \item $\Autann{\IQ} = \set{\id}$
  \item $\Autann{\IZ[i]} = \set{\id, \text{conj. complexe}}$
  \item $\Autann{\IC}$ non dénombrable
  \item $\Autann{M_n(\IZ)=\set{X \mapsto UXU^{-1};U \in GL_n(\IZ)}}$
\end{itemize}

\subsection{Exemples de morphismes canoniques}
\begin{enumerate}
  \item $T$ sous anneau de $R$. L'inclusion
    $\begin{array}[t]{r@{\ }l}
      \iota_T : T &\to R\\
      t &\mapsto t
    \end{array}$ est un morphisme d'anneau injectif.\\
    Dire $\iota_T$ est un morphisme d'anneau est équivalent à dire $T$ sous
    anneau de $R$.
  \item $I$ ensemble, $S=\prod_{i\in I}S_i$, $S_i$ anneaux.
    $\forall j \in I, \begin{array}[t]{r@{\ }l}
      p_j : S &\surj S_j \\
      (S_i)_{i\in I} &\mapsto S_j
    \end{array}$ est un morphisme d'anneau surjectif.
\end{enumerate}

\paragraph{Remarque}
$\begin{array}[t]{r@{\ }l}
  S_j &\to \prod_{i\in I} S_i\\
  s &\mapsto (r_i)_{i \in I}, (r_i = 0 \text{ si } i\neq j, r_i=s
  \text{ si } i=j)
\end{array}$ est un morphisme de groupe mais pas un morphisme d'anneau, sauf
si $S_i=\set{0}$ pour $i\neq j$.

\chapter{Anneaux commutatifs}
À partir de ce chapitre, tous les anneaux $A$ considérés sont commutatifs.

\section{Polynômes}
\subsection{Définitions}
$A$ anneau commutatif non nul.
\subsubsection{$A[X]$}
\[A[X] = \set{\sum_{n \in \IN} a_n X^n; a_n = 0_A \ \text{p.p}}\]

Loi $+$ sur $A[X]$ : $\displaystyle \left( \sum_{n \in \IN} a_n X^n \right) +
\left( \sum_{n \in \IN} b_n X^n \right) := \sum_{n \in \IN} (a_n + b_n) X^n$

Alors $(A[X], +)$ est un groupe abélien.
\paragraph{Remarque}
$\begin{array}[t]{r@{\ }l}
  (A[X], + ) &\inj \left(\prod\limits_{n \in \IN}, +\right)\\
  \sum\limits_{n\in \IN} a_n X^n &\mapsto (a_0, a_1, \dots, a_n, \dots)
\end{array}$ injection de groupe.

Loi $.$ sur $A[X]$ : $\displaystyle \left( \sum_{n \in \IN} a_n X^n \right) .
\left( \sum_{n\in \IN} b_n X^n \right) := \sum_{n \in \IN} c_n X^n$ avec
$\displaystyle c_n = \sum_{k+m = n} a_m b_k =\sum_{k+m = n} b_k a_m$

Neutre pour $.$ : $\displaystyle 1_A X^0 + \sum_{n \geq 1} 0_A X^n =
1_{A[X]}$.

Alors $\left( A[X], +, . \right)$ est un anneau commutatif.

\paragraph{Remarque} $ (a_n)_{n \in \IN}, (b_n)_{n \in \IN} \in
\prod_{n \in \IN} A$.\\
$(a_n) * (b_n) = (c_n)$ où $\displaystyle c_n = \sum_{k+m} a_k b_m$ (produit
de convolution)

Alors $\displaystyle\left( \prod_{n\in \IN} A , +, * \right)$ est un anneau
commutatif.

\subsubsection{Degré d'un polynôme}
Soit $P(X)=\sum_{n \in \IN } a_n X^n \in A[X]$ non nul $\so$ il existe un plus
grand $d$ tq $a_d \neq 0$.

Alors $d=\deg P$

\paragraph{Convention} $\deg 0_{A[X]} = -\infty$

On a:
\begin{eqnarray*}
\deg (P+Q) &\leq& \max \set{\deg P, \deg Q}\\
\deg (P.Q) &\leq& \deg P + \deg Q
\end{eqnarray*}

\subsubsection{$A[X, Y]$ et généralisation}
\[A[X, Y] = \set{\sum_{(n,m) \in \IN \times \IN} a_{n,m} X^n Y^m; a_{n,m} = 0
\ \text{p.p}}\]

\[\sum_{(n,m) \in \IN \times \IN} a_{n,m} X^n Y^m + \sum_{(n_,m) \in \IN
\times \IN} b_{n,m} X^n Y^m = \sum_{(n,m) \in \IN \times \IN} (a_{n,m} +
b_{n,m}) X^n Y^m\]
\begin{multline*}
\left(\sum_{(n,m) \in \IN \times \IN} a_{n,m} X^n Y^m\right)
. \left(\sum_{(n_,m) \in \IN \times \IN} b_{n,m} X^n Y^m\right) = \sum_{(n,m)
\in \IN \times \IN} c_{n,m} X^n Y^m ; \\
c_{n,m} = \sum_{\substack{ n_1+n_2 =n\\
  m_1 + m_2 = m} } a_{n_1,m_1} . b_{n_2, m_2}
\end{multline*}

Alors $\left( A[X,Y], +, . \right)$ est un anneau commutatif.

\paragraph{Généralisation} $(A[X_1, \dots, X_n], +, .)$ anneau commutatif.\\
$(A[X_i; i\in \IN], +, .)$ anneau commutatif.

\subsection{Morphismes}
\subsubsection{Injection}
$\begin{array}[t]{r@{\ }l}
  A  &\inj A[X]\\
  a &\mapsto a X^0
\end{array}$
morphisme d'anneau injectif. Permet d'identifier $A$ à un sous anneau de
$A[X]$ (les polynômes constants).

Aussi, $
\begin{array}[t]{r@{\ }l}
  A[X] &\inj A[X,Y]\\
  \sum\limits_{n\in \IN} a_n X^n &\mapsto  \sum\limits_{n \in \IN} a_n
  X^n Y^0
\end{array}$ morphisme d'anneau injectif.

\subsubsection{Evaluation}
Soit $Q(X) = \sum b_m X^m \in A[X]$.
$\begin{array}[t]{r@{\ }l}
  C_Q : A[X] &\to A[X]\\
  P(X) &\mapsto P(Q(X))\\
  \sum a_n X^n &\mapsto \sum a_n \left( \sum b_m X^m \right)^n
\end{array}$ morphisme d'anneau.

\paragraph{Deux cas particuliers importants}
\begin{enumerate}
  \item Soit $a\in A$, on prend $Q(X) = X + a$.
    $\begin{array}[t]{r@{\ }l}
      C_{X+a} : A[X] &\to A[X]\\
      P(X) &\mapsto P(X+a)
    \end{array}$ automorphisme d'anneau. $C_{X+a}^{-1}=C_{X-a}$
  \item Soit $a\in A_i$; on prend $Q(X)=a$.\\
    $\begin{array}[t]{r@{\ }l}
      C_a: A[X] &\to A[X]\\
      P(X) &\mapsto P(a) \in A
    \end{array}$\\
    $\begin{array}[t]{r@{\ }l}
       \ev{a}: A[X] &\to A\\
      P(X) &\mapsto P(a)
    \end{array}$ morphisme d'anneau surjectif.
\end{enumerate} 
\subsubsection{Lemme : $\ker(\ev a)$}
\[\ker(\ev a) = \set{(X-a).R(X); R(X) \in A[X]}\]
\begin{comment}
preuve 07/10/09 p1
\end{comment}

\subsubsection{Changement d'anneau de coefficients}
$f:A \to B$ morphisme d'anneau. Alors $
\begin{array}[t]{r@{\ }l}
  F: A[X] &\to B[X]\\
  \sum a_n X^n &\mapsto \sum f(a_n) X^n
\end{array}$ est un morphisme d'anneau. Et si $f$ est iso, alors $F$ l'est
également.

\subsubsection{Propriété universelle de $A[X]$}
$\forall f : A \to B$ morphisme d'anneau, $\forall b \in B$, $\exists !
\mt{f} : A[X] \to B$ morphisme d'anneau tel que $\rstrct{\mt{f}}{A} = f$ et
$\mt{f}(X)=b$ 
\begin{comment}
  \pfootnote{Preuve 07/10/09 p1 verso}
\end{comment}

\paragraph{Remarque} $\im \left( \begin{array}{r@{\ }l}
  A[X] &\inj B[X] \to B \\
  P(X) &\xmapsto{\hphantom{\!\textstyle \!\inj B[X] \to\!\!}}  P(b)
\end{array}\right):=A[b]=\gen{A\cup\set{b}}_B$\\
$\im\left( \IZ[X] \inj \IC[X] \xrightarrow{\ev i} \IC\right)=\IZ[i]$

\paragraph{Conséquences}
\begin{enumerate}
  \item $f:A[X] \inj A[X,Y]$ morphisme d'anneau injectif\\
    $\soo \begin{array}[t]{r@{\ }l}
      \left( A[X] \right)[Y] &\bij A[X,Y]\\
      \sum\limits_{m\geq 0} \left( \sum\limits_{n\geq 0} a_{m,n} X^n \right)
      Y^m &\mapsto \sum\limits_{m,n \geq 0} a_{m,n} X^n Y^m
    \end{array}$
  \item $\mathcal{F}(A,A)=\set{\text{appl. } A \to A}$ anneau.

    $\begin{array}[t]{r@{\ }l}
      \varphi : A[X] &\to \mathcal{F}(A,A)\\
      P(X)=\sum\limits_{0\leq k \leq n} a_k X^k &\mapsto
      \begin{array}[t]{r@{\ }l}
        \varphi(P) : A &\to A\\
        a &\mapsto \sum\limits_{0\leq k \leq n} a_k a^k = P(a)
      \end{array}
    \end{array}$\\
    $\im \varphi = \set{\text{appl. polynômiales} A \to A}$ sous anneau de
    $\mathcal F(A,A)$.\\
    $\ker \varphi \neq \set{0}$ en général.
\end{enumerate}

\subsection{Division euclidienne}
\subsubsection{Proposition, existence et unicité du quotient et reste}
Soient $P(X), S(X) \in A[X]$ avec $P(X) \neq 0$ et $S(X)=b_m X^m +
b_{m-1} X^{m-1} + \dots + b_1 X + b_0$ avec $m\geq 0$ et $b_m \in A^\times$.

Alors $\exists ! (Q(X), R(X)) \in A[X] \times A[X]$ tel que
\[P(X) = Q(X) . S(X) + R(X)\]
et $\deg R < \deg S = m$.

\begin{comment}
  Preuve 07/10/09 p2 verso
\end{comment}

\paragraph{Remarques}
\begin{itemize}
  \item $b_m \in A^\times \so S(X) \neq 0$
  \item Si $A$ corps, alors $b_m \in A^\times \iff S(X)\neq 0$
  \item $b_m \in A^\times \so \forall Q(X) \in A[X], \deg(Q.S) = \deg Q + \deg
    S$
\end{itemize}

\section{Quotients}
\subsection{Anneau quotient}
\begin{comment}
  Détails, diagramme, 08/10/09 p1 verso.
\end{comment}
\subsubsection{Idéal}
Un idéal de $A$ est un sous groupe $(I,+)$ de $(A,+)$ tel que
\[\forall a \in A, \forall c\in I, a.c \in I\]
\paragraph{Exemples}
\begin{itemize}
  \item $\set{0_A}$ et $A$ sont des idéaux de $A$.
  \item $f:A\to B$ morphisme d'anneaux, alors $\ker f$ idéal de $A$.
  \item $c\in A; \set{a.c; a\in A} = A.c = c.A \overset{\text{not}}{=}(c)$ est
    un idéal de $A$.
  \item $\set{(0_A, b); b \in B}$ et $\set{(a,0_B); a \in A}$ sont des idéaux
    de $A\times B$.
  \item Les idéaux de $\IZ$ sont les $n\IZ$ où $n\in \IZ$.
\end{itemize}

\subsubsection{Définition : anneau quotient}
Soit $(I,+) < (A,+)$.\\
La 2\ieme loi sur $A$ induit, via $\begin{array}[t]{r@{\ }l}
  \pi_I : A &\surj A/I\\
  a &\mapsto a+I
\end{array}$ une loi sur $A/I$ ssi $I$ est
un idéal de $A$.

Dans ce cas, $(A/I, +, .)$ est un anneau commutatif, appelé anneau quotient de
$A$ par $I$.


\subsubsection{La projection $\pi_I$ est un morphisme}
Si $I$ idéal de $A$, alors $\pi_I: A \to A/I$ est un morphisme d'anneau.

\subsubsection{$(A/I)^\times$}
$(A/I)^\times = \set{a + I \in A/I \Such \exists b \in A, ab - 1_A \in I}$

\subsubsection{Exemples}
\begin{itemize}
  \item $A=\IZ; I=n\IZ, n \in \IZ \so (\IZ/n\IZ, +, .)$ anneau commutatif fini
    si $n\neq 0$ et non nul si $n\neq \pm 1$.
  
    $(\IZ/n\IZ)^\times = \set{a+n\IZ \such \pgcd(a,n)=1}$ par Bézout.

  \item $A=\IZ[i], n \in \IZ; I=n\IZ[i]$.

    $A/I=\IZ[i]/n\IZ[i]$ anneau commutatif fini (d'ordre $n^2$).

  \item $A=\IR[X]; n \in \IN, I_n=X^n.\IR[X]$ idéal.

    Alors $A/I_n=\IR[X]/I_n$ anneau commutatif.\\
    Soit $x=\pi_{I_n}(X)$; alors on a :\\
    $A/I_n=\set{a_0+a_1 x + \dots + a_{n-1} x^{n-1}; a_i \in \IR}$ et $x^n =
    0_{A/I_n}$.
\end{itemize}

\subsubsection{Idéaux d'anneau quotient}
Soit $J$ idéal de $A$ tel que $I\subseteq J$. En particulier, $(I,+) <
(J,+)$ et l'anneau quotient $J/I$ est un idéal de $A/I$.

\subsubsection{La projection $\pi_I$ induit une bijection croissante}
La projection $\pi_I : A \surj A/I$ induit une bijection croissante pour
$\subseteq$.
\[
\begin{array}[t]{r@{\ }l}
  \set{\text{idéaux de }A\text{ contenant } I} \longrightarrow&
  \set{\text{idéaux de }A/I}\\
  J \xmapsto{\hphantom{\textstyle \text{ contenant } I}}& J/I
\end{array}
\]

\paragraph{Exemple d'application}
Soit $n\neq 0 \in \IZ$. Les idéaux de $\IZ/n\IZ$ sont en bijection $\nearrow$
avec les $m\IZ, m \in \IZ$ tels que $n\IZ \subseteq m\IZ \iff m \mid n$.

\subsection{Morphismes}
$f : A \to B$. Soit $J$ un idéal de $B$, $\begin{array}[t]{r@{\ }l}
  \pi_J : B &\surj B/J\\
  b &\mapsto b+ J
\end{array}$ morphisme d'anneau. $\xymatrix{
A \ar[dr]_{\substack{\pi_J \circ f\\ \text{morph.}\\ \text{d'ann}}}
\ar[r]^f & B \ar@{->>}[d]^{\pi_J}\\
  & B/J
}$

\subsubsection{Factorisation de morphisme d'anneau $f$ par $\pi_I$ ssi $I
\subseteq \ker f$}
Soit $f:A\to B$ morphisme d'anneau. $I$ idéal de $A$\\
$f$ se factorise par $\pi_I:A\surj A/I$, c'est-à-dire $\exists
\overline{f}:A/I \to B$ tel que $f=\overline f \circ \pi_I$\\
ssi $I\subseteq \ker f$.

Dans ce cas, $
\begin{array}[t]{l}
  \im \overline{f}= \im f\\
  \ker \overline f = \ker f / I
\end{array}$\hspace{4.2em}
$\xymatrix{
  c\in I\ar@{|->}[dd]_{\pi_I} &A \ar[r]^f \ar@{->>}[d]_{\pi_I} & B & 0_B\\
  &A/I \ar@{.>}[ur]_{\overline{f}} &\\
  0_{A/I} \ar@/_2pc/@{|->}[uurrr]_{\overline f}
}$
\paragraph{Cas particulier $I=\ker f$ :}
$\exists !$ morphisme d'anneau $\begin{array}[t]{r@{\ }l}
  A/\ker f &\xhookrightarrow{\overline f} B \\
  a+ \ker f &\mapsto f(a)
\end{array}$ factorisant $f$ et $\overline f$ injectif.

\paragraph{Exemples}
\begin{enumerate}
  \item $A=\IZ, B=\IZ/n\IZ, f = \pi_{n\IZ}:=\pi_n, I = m\IZ$
    \hspace{4.2em}$\xymatrix{
    \IZ \ar@{->>}[r]^{\pi_n} \ar@{->>}[d]_{\pi_m} & \IZ/n\IZ\\
    \IZ/m\IZ \ar@{.>}[ur]_{\overline{\pi_n}}
    }$\\Il existe un morphisme d'anneau $\overline{\pi_n}: \IZ/m\IZ \to
    \IZ/n\IZ$ tel que $\pi_n=\overline{\pi_n} \circ \pi_m$ ssi $m\IZ \subseteq
    n\IZ = \ker \pi_n \iff $ $n$ divise $m$.

    Dans ce cas, $\begin{array}[t]{r@{\ }l}
      \overline{\pi_n} : \IZ/m\IZ &\to \IZ/n\IZ\\
      a+m\IZ &\mapsto a+n\IZ
    \end{array}$ est un morphisme d'anneau $\iff n \mid m$.\\
    Et $\im \overline{\pi_n}= \im \pi_n=\IZ/n\IZ$ et $\ker
    \overline{\pi_n}=\ker \pi_n/\ker \pi_m=n\IZ/m\IZ$.
    
    De plus, on obtient un isomorphisme d'anneau :
    \[\overline{\overline{\pi_n}} : \left( \IZ/_{m\IZ} \right)/_{\left(
    n\IZ/m\IZ
    \right)} \bij \IZ/n\IZ\]
  \begin{comment}
  \item Exemple avec $A\times B / \set{(a,0_B)}$ 15/10/09 p2
  \end{comment}
  \item $
    \begin{array}[t]{r@{\ }l}
      f=\ev a : A[X] &\surj A\\
      P(X) &\mapsto P(a)
    \end{array}
    $ morphisme d'anneau surjectif. \\$\ker(\ev a) = \set{(X-a)Q(X); Q(X) \in
    A[X]} \overset{\text{not}}{=} (X-a)$.\\
    D'où $
    \begin{array}[t]{r@{\ }l}
      \overline{\ev a} : A[X] / (X-a) &\bij A \\
      P(X) \mod (X-a) &\mapsto P(a)
    \end{array}
       $ isomorphisme d'anneau.
  \begin{comment}
  \item Injection, $f^{-1}$ et création de sous anneau de $\IC$ 15/10/09 p2
    verso.
  \end{comment}
\end{enumerate}

\section{Idéaux}
$A$ anneau commutatif, $\mathcal{I}(A) = \set{\text{idéaux de }A}$. On a
toujours $\set{0_A} \in \mathcal{I}(A)$ et $A \in \mathcal{I}(A)$.
\subsection{Idéaux inversibles}
\subsubsection{Lemme : intersection entre un idéal et les inversibles}
$I \in \mathcal{I}(A), A^\times = \set{\text{inversibles de } A}$
\[I\cap A^\times \neq \emptyset \iff A=I\]

\begin{comment}
  Preuve 22/10/09 p1
\end{comment}

\paragraph{Conséquence} Si $I\in \mathcal{A}$ et $I$ sous anneau, alors $1_A
\in I$ et donc $I=A$.

\subsubsection{Corollaire du lemme précédant : idéaux de corps}
$A$ corps $\iff \mathcal{I}(A) = \set{\set{0_A}, A}$

\paragraph{Remarque} $K, L$ corps, $K\times L$ n'est jamais un corps car il
existe toujours deux idéaux non triviaux : $\set{0_K}\times L$ et
$K\times \set{0_L}$

\subsection{Opérations sur les idéaux}
\subsubsection{Intersection}
Soit $E$ ensemble d'indices quelconques.
\[\forall i \in E, I_i \in \mathcal{I}(A) \so \bigcap_{i\in E} I_i \in
\mathcal{I}(A)\]
\begin{comment}
  Preuve 22/10/09 p2
\end{comment}

\paragraph{Remarque}
À l'inverse, une union d'idéaux n'est pas nécéssairement un idéal.

\subsubsection{Lemme/définition : idéal engendré par un ensemble}
Soit $P\neq \emptyset \subseteq A$, il existe un plus petit (pour $\subseteq$)
idéal contenant $P$, on le note $(P)$ : l'idéal engendré par $P$.
\begin{comment}
  Preuve 22/10/09 p2
\end{comment}

\[(\underb{P}{\set{a_1, \dots, a_n}})\tnot{=}(a_1, \dots, a_n) =
\set{\sum_{i=1}^n b_i a_i \Such b_i \in A}\]
$(\set{a}) = (a) = a. A$ est dit idéal principal.

\subsubsection{Somme}
$I, J \in \mathcal I (A)$
\[I+J:=(I\cup J)=\set{x+y \Such x \in I, y \in J}\]
\[(a_1)+(a_2)+\dots+(a_n) = (a_1, a_2, \dots, a_n)\]

\subsubsection{Produit}
$I,J \in \mathcal{I}(A)$
\begin{align*}
  I.J&:=\set{\sum_{\text{finie}}x_i y_i \Such x_i \in I, y_i \in J}\\
  &=\left( \set{xy \such x\in I, y\in J} \right)\\
  &=\gen{xy \such x\in I, y\in J}_{\text{grp }+}
\end{align*}

\paragraph{Remarque} $I.J \subseteq I\cap J$

\subsection{Théorème Chinois}
\subsubsection{Cas de deux idéeaux}
Soit $A$ anneau commutatif. Soient $I, J$ idéaux de $A$.\\
Soit $\begin{array}[t]{r@{\ }l}
  \varphi : A &\to A/I \times A/J\\
  a &\mapsto (a \mod I, a \mod J)
\end{array}$ morphisme d'anneaux.
\begin{enumerate}[(i)]
  \item $\ker \varphi = I \cap J$
  \item $\varphi$ surjective ssi $I+J=A$
\end{enumerate}
\begin{comment}
  Preuve 29/10/08 p1
\end{comment}

\subsubsection{Cas particulier de $\IZ$}
$\varphi: \IZ \to \IZ/n\IZ \times \IZ/m\IZ$
\begin{enumerate}[1)]
  \item $\ker \varphi = n\IZ \cap m\IZ = \ppcm(n,m)\IZ$
  \item $\varphi$ surjective $\iff\underb{\pgcd(n,m)\IZ}{n\IZ+m\IZ}=\IZ \iff
    \pgcd(n,m)=1 \iff \ppcm(n,m)=n.m$\\
    $\soo$ (factorisation) $\IZ/nm\IZ \bij \IZ/n\IZ \times \IZ/m\IZ$
\end{enumerate}


\subsubsection{Généralisation}
$I_1, \dots, I_n$ idéaux de $A$ tels que $\forall i \neq j, I_i + I_j = A$\\
Alors $
\begin{array}[t]{r@{\ }l}
  A/\bigcap_{1\leq i \leq n} I_i &\displaystyle
  \overset{\sim}{\longrightarrow} \prod_{1 \leq i\leq n} A/I_i\\
  \displaystyle a+\bigcap_{1\leq i \leq n} I_i &\displaystyle \longmapsto
  \left( a+I_i \right)_{1\leq i \leq n}
\end{array}
$

\subsubsection{Fonction indicatrice d'Euler}
$n\geq 2; \varphi(n):=|(\IZ/n\IZ)^\times|$\\
$\varphi:\IN_{\geq 2} \longrightarrow \IN$ fonction indicatrice d'Euler.
On a, si $n=p_1^{m_1}. \dots . p_r^{m_r}$, décomposition de $n$ en facteurs
premiers :
\[\varphi(n)=p_1^{m_1-1}(p_1 - 1). \dots . p_r^{m_r-1}(p_r-1)\]
\begin{comment}
  Preuve Application du thm chinois généralisé, 29/10/08 p2
\end{comment}

\subsection{Idéaux premiers}
\subsubsection{Anneaux intègres}
Un anneau intègre $A$ est intègre si $A$ non nul et si\\
$\forall a,b \in A, a.b = 0 \soo a=0$ ou $b=0$.

\chapter{Arithmétique des anneaux}
\end{document}
% vim:expandtab:shiftwidth=2:nu
